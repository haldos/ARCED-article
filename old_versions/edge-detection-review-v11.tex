%-------------------------------------------------------------------------------
% A Review of Classic Edge Detectors - v10
% by Haldo Spontón & Juan Cardelino
% IPOL 2012
%-------------------------------------------------------------------------------

\documentclass{ipol}
\ipolSetYear{2012}
\ipolSetMonth{06}
\ipolSetDay{06}
%\ipolSetID{arced}
\ipolSetTitle{A Review of Classic Edge Detectors}
\ipolSetAuthors{Haldo Spont\'on$^1$, Juan Cardelino$^2$, Rafael Grompone von Gioi$^3$}
\ipolSetAffiliations{%
$^1$ IIE, UdelaR, Uruguay (\texttt{haldos@fing.edu.uy})\\
$^2$ IIE, UdelaR, Uruguay (\texttt{juanc@fing.edu.uy})\\
$^3$ CMLA, ENS Cachan, France (\texttt{grompone@cmla.ens-cachan.fr})}

\usepackage{hyperref,verbatim,graphicx,amsmath,amssymb,amssymb,dsfont}
\usepackage[ruled,linesnumbered]{algorithm2e}
\usepackage{sidecap}
\usepackage[hang,center]{subfigure}
\usepackage[table]{xcolor}
\usepackage{multicol}
\usepackage{listings}
\usepackage{color}
\usepackage{calc}
\usepackage{array}
%\usepackage{algorithm}
\usepackage{algorithmic}
\newtheorem{theorem}{Theorem}
\numberwithin{equation}{section}
\numberwithin{table}{section}
%\numberwithin{figure}{section}
\begin{document}
\bibliographystyle{ieeetr}
\lstset{
	backgroundcolor=\color[rgb]{0.94,0.94,0.94},
    tabsize=4,
    inputencoding=utf8x,
         extendedchars=\true,
         language=C,
         basicstyle=\scriptsize,
                 showtabs=false,
        showspaces=false,
        showstringspaces=false,
        identifierstyle=\ttfamily,
        keywordstyle=\color[rgb]{0,0,1},
        commentstyle=\color[rgb]{0.133,0.545,0.133},
        stringstyle=\color[rgb]{0.627,0.126,0.941},
}
% Square cells table
\newlength\celldim \newlength\fontheight \newlength\extraheight
\newcounter{sqcolumns}
\newcolumntype{S}{ @{}
  >{\centering \rule[-0.5\extraheight]{0pt}{\fontheight + \extraheight}}
  p{\celldim} @{} }
\newcolumntype{Z}{ @{} >{\centering} p{\celldim} @{} }
\newenvironment{squarecells}[1]
  {\setlength\celldim{2em}%
   \settoheight\fontheight{A}%
   \setlength\extraheight{\celldim - \fontheight}%
   \setcounter{sqcolumns}{#1 - 1}%
   \begin{tabular}{|S|*{\value{sqcolumns}}{Z|}}\hline}
% squarecells tabular goes here
  {\end{tabular}}
\newcommand\nline{\tabularnewline\hline}

%-------------------------------------------------------------------------------

\begin{abstract}
In this paper some of the classic alternatives for edge detection in digital images are studied. The main idea 
behind edge detection is to find where abrupt changes in the intensity of an image have occurred. 
The first family of algorithms reviewed in this work uses the first derivative to find the changes of intensity, 
such as Sobel, Prewitt and Roberts. In the second reviewed family, second derivatives are used, for example in algorithms 
like Marr-Hildreth and Haralick. \\
Results obtained from a qualitative point of view (perceptual) and from a quantitative 
point of view (number of operations, execution time) are compared, considering different ways to convolve an 
image with a kernel (step required in some of the algorithms).
\end{abstract}

%-------------------------------------------------------------------------------

% No mas link, mirar como se pone en las publicaciones nuevas. (ir a buscar el codigo a la pagina de la publicacion)
% lo mismo con la demo... sacar GIT

%% VER SI HAY UN TEXTO ESTANDAR PARA ESTO
\begin{ipolCode}
For all the algorithms reviewed, an open source C implementation is provided that can be downloaded from 
the IPOL publication of this article. An 
\href{http://dev.ipol.im/~haldos/ipol_demo/xxx_edges/}{online demonstration} is also available, where 
you can test and reproduce our results.
\end{ipolCode}

% You can also follow this code project in this \href{git://github.com/haldos/edges.git}{GIT repository} 
% \href{http://iie.fing.edu.uy/~haldos/downloads/edge_detectors_v0.1.tar.gz}{here}

%%-------------------------------------------------------------------------------

%\begin{ipolSupp}
%\end{ipolSupp}

%-------------------------------------------------------------------------------

\section{Introduction}
\label{sec:intro}

% cambiar lo de segmenting images. Decir que edge detection es una de las operaciones
% basicas y mas antiguas en el procesamiento de imagenes.
% Edge detection is one of the first approaches published in the literature for 
% segmenting images. 
The basic idea of edge detection algorithms is to detect abrupt changes in image intensity. 
Detecting those changes in intensity for the purpose of finding edges in images 
can be accomplished using first or second order derivatives. 

In the 70's, edge detection methods were based on using small operators 
(such as Sobel masks), attempting to compute an approximation of the
first derivative of the image \cite{Gonzalez2007Digital}. Next section describes such algorithms and serves as an introduction to a more sophisticated analysis of the edge detection process.

% en este parrafo, puede ser cambiar trough por valley
In 1980 Marr and Hildreth \cite{AIM-518} argued that intensity changes are not independent 
of image scale, so edge detection requires the use of different size 
operators. They also argued that a sudden intensity change will be seen 
as a peak (or trough) in the first derivative or, equivalently, as a zero 
crossing in the second derivative. This algorithm is presented in Section \ref{sec:second}. 
Haralick's algorithm \cite{bb20239} is an alternative approach based on the second derivative 
which is also reviewed in Section \ref{sec:second}. This algorithm has the particularity of 
proposing a model to locally approximate the image around a point; then, using this model, 
an approximation to the second derivative of the image can be calculated analytically and 
finding edges is achieved by imposing a condition over the model parameters. 

Along with this paper, a detailed and well commented source code is presented, which 
implements the described algorithms. In section \ref{sec:appendix1} some common mathematical developments are presented. 
%A commented explanation of the source code can be found in section \ref{sec:appendix2}. This work also provides tools for code documentation 
%(see section \ref{sec:appendix2}). 
Results of the implemented algorithms are presented in Section \ref{sec:results}, along with examples 
to compare their performance. Conclusions are detailed in Section \ref{sec:conclusions}.

\nocite{IPOL}

%-------------------------------------------------------------------------------

\section{Algorithms based on first derivative}
\label{sec:first}

Algorithms based on first derivative studied in this paper have a common scheme, with 
the only difference in the type of filtering. Figure \ref{fig:blockdiagram1} shows
a block diagram of these algorithms. \\

\begin{figure}[!h]
	\centering
	\includegraphics[width=\textwidth]{blockdiagram1.pdf}
	\caption{Block diagram of first derivative edge detection algorithms.}
	\label{fig:blockdiagram1}
\end{figure}

The usual tool to find the amplitude and direction of changes in 
intensity of an image $f$ is the gradient operator (denoted as $\nabla$), defined 
as the vector
\begin{equation}
	\nabla f = 
				\begin{bmatrix} 
					f_x \\ f_y
				\end{bmatrix}
	=				
				\begin{bmatrix} 
					\cfrac{\partial f}{\partial x} \\ \cfrac{\partial f}{\partial y} \\
				\end{bmatrix}.
\end{equation}

The magnitude ($M$) and direction ($\alpha$) of the gradient vector $\nabla f$ at location $(x,y)$
are calculated as
\begin{equation}
\label{eq:mag_alpha}
	\begin{cases}
		M(x,y) = \sqrt{f_x^2 + f_y^2} \\
		\alpha(x,y) = \tan^{-1} \left( \dfrac{f_y}{f_x} \right) .
	\end{cases}
\end{equation}

The direction of an edge at an arbitrary location $(x,y)$ of the image is 
orthogonal to the direction $\alpha(x,y)$ of the gradient vector.

To obtain the gradient, the partial derivatives $\partial f/\partial x$ and $\partial f/\partial y$ 
need to be computed at every pixel in the image. When dealing with digital images, numerical approximations 
of the partial derivatives are required, calculated in a neighborhood of each point. In the following,
the methods of Roberts, Prewitt and Sobel will be studied, whose main difference is how they perform this calculation. 

%-------------------------------------------------------------------------------

\subsection{The \textit{Roberts} operators}

The most usual approach to approximate the first derivative is to take the Taylor expansion of first order with
a small $h$. Thus $f'(x)$ is computed as
\begin{equation}
	f'(x) \simeq \frac{f(x+h) - f(x)}{h}.
\end{equation}
% explicar que esta pasando con el h
% en las operaciones de abajo, dejar explicito que se considera h=1
The image is composed by discrete points of coordinates $(i,j)$, so taking $x=ih$ and $y=jh$, the 
first derivative of the image is approximated, based on the intensity values ​​at points of the image, as\footnote{This approximation considerates $h=1$, so the partial derivatives can be computed as intensity difference between neighbor pixels.}:
\begin{equation}
\label{eq:roberts1}
	f_x = \frac{\partial f(x,y)}{\partial x} \cong f(i+1,j) - f(i,j)
\end{equation}
and
\begin{equation}
\label{eq:roberts2}
	f_y = \frac{\partial f(x,y)}{\partial y} \cong f(i,j+1) - f(i,j).
\end{equation}

Equations \ref{eq:roberts1} and \ref{eq:roberts2} can be implemented for all values of $x$ and $y$
by filtering the image $f(x,y)$ with the 1-D masks in Figure \ref{fig:1dmasks}. %But when finding 
%diagonal edges is desired, 2-D masks are needed. 
The \textit{Roberts operators} are one of the earliest 
attempts to use 2-D masks for this purpose. These operators are based on computing the diagonal 
diferences implemented by filtering an image with the masks in Figure \ref{fig:roberts}. By convention 
in these figures, x-coordinates grow from left to right and y-coordinates grow top-down. It is also 
usual to scale the mask values, in order to have unit $L_2$ norm. 

% la primer figura es una forma simple de calcular las derivadas, pero roberst, en [referenfcia]
% propone calcularlos como en la segunda figura. Nota: cambiar signos.

%% VER NORMALIZACION

\begin{SCfigure}[2][!h]
	\centering
	\begin{squarecells}{1}
		-1   \nline
		1  \nline
	\end{squarecells}
	\quad
	\begin{squarecells}{2}
		-1 & 1   \nline
	\end{squarecells}
	\caption{One-dimensional normalized masks used to implement equations \ref{eq:roberts1} and \ref{eq:roberts2}.}
	\label{fig:1dmasks}
\end{SCfigure}

\begin{SCfigure}[1][!h]
	\centering
	\begin{squarecells}{2}
		-1 &  0  \nline
		0 & 1  \nline
	\end{squarecells}
	\quad
	\begin{squarecells}{2}
		0  & -1  \nline
		1 & 0  \nline
	\end{squarecells}
	\caption{Normalized \textit{Roberts cross-gradient} 2-D masks.}
	\label{fig:roberts}
\end{SCfigure}

Figure \ref{fig:1dmasks} shows the simplest way to compute derivatives, but Roberts in his article proposes
to compute them as shown in Figure \ref{fig:roberts}. In order to use the same convolution code for all algorithms, our implementation uses $3\times3$ matrices, adding a row and a column of zeros to the matrices in Figure \ref{fig:roberts}, what produces the same result.

%-------------------------------------------------------------------------------

\subsection{The \textit{Prewitt} operators}

% agregar referencia

Masks of size $2\times2$, although conceptually simple, are not symmetrical with respect to the central points. 
Having symmetrical edges is a desirable property and can only be achieved with oddly sized masks, the smallest 
of them being the 3x3. These masks provide more information to find the direction of the edges, because they take 
into account information on opposite sides of the central point. 

The simplest digital approximation to the partial derivatives using masks of size $3\times3$ are obtained 
by taking the difference between the third and first rows (or columns) of the $3\times3$ region. The
difference between the third and first rows approximates the derivative in the x-direction, and 
the difference between the third and first columns approximates the derivative in the y-direction.
These approximations can be implemented by filtering the image with the two masks in Figure \ref{fig:prewitt}.
These masks are called \textit{Prewitt operators}. 

\begin{SCfigure}[1][!h]
	\centering
	\begin{squarecells}{3}
		$-\frac{1}{6}$ 	& $-\frac{1}{6}$ 	& $-\frac{1}{6}$	\nline
		0 			& 0			& 0			\nline
		$\frac{1}{6}$ 	& $\frac{1}{6}$ 	& $\frac{1}{6}$	\nline
	\end{squarecells}
	\quad
	\begin{squarecells}{3}
		$-\frac{1}{6}$ 	& 0 	& $\frac{1}{6}$	\nline
		$-\frac{1}{6}$	& 0	& $\frac{1}{6}$	\nline
		$-\frac{1}{6}$ 	& 0 	& $\frac{1}{6}$	\nline
	\end{squarecells}
	\caption{Normalized \textit{Prewitt} 2-D masks of size $3\times3$.}
	\label{fig:prewitt}
\end{SCfigure}

These masks can be obtained (beyond normalization) as the convolution of a horizontal derivation mask 
$[-1\ 0\ 1]$ with a vertically moving average $[1\ 1\ 1]^T$, and vice versa. Hence, these operators have 
smoothing properties in the direction orthogonal to the gradient.

%-------------------------------------------------------------------------------

\subsection{The \textit{Sobel} operators}

A slight variation of the Prewitt operators uses more weight on the central coefficients of the 
difference. It can be shown that using double weight in the center location provides better image smoothing. This
variation is implemented using masks in Figure \ref{fig:sobel}. These operators are called 
\textit{Sobel operators}. 

Sobel masks can be seen (beyond normalization) as the convolution of a horizontal derivation mask 
$[-1\ 0\ 1]$ with a vertical smoothing filter $[1\ 2\ 1]^T$ (closer to a Gaussian response than Prewitt), and vice versa. 
Hence, these operators have better smoothing properties, as mentioned in the preceding paragraph. 

\begin{SCfigure}[1][!h]
	\centering
	\begin{squarecells}{3}
		$-\frac{1}{8}$ 	& $-\frac{1}{4}$ 	& $-\frac{1}{8}$	\nline
		0 			& 0			& 0			\nline
		$\frac{1}{8}$ 	& $\frac{1}{4}$ 	& $\frac{1}{8}$	\nline
	\end{squarecells}
	\quad
	\begin{squarecells}{3}approximate
		$-\frac{1}{8}$ 	& 0 	& $\frac{1}{8}$	\nline
		$-\frac{1}{4}$	& 0	& $\frac{1}{4}$	\nline
		$-\frac{1}{8}$ 	& 0 	& $\frac{1}{8}$	\nline
	\end{squarecells}
	\caption{Normalized \textit{Sobel} 2-D masks of size $3\times3$.}
	\label{fig:sobel}
\end{SCfigure}

The computational difference between Prewitt and Sobel masks is not an issue. It is preferable to 
use Sobel operators, because they are better localized, and their filter are also less aliased, 
because of the weighted shape of $[1\ 2\ 1]$.

%-------------------------------------------------------------------------------

\subsection{Computation of the edges}

As mentioned before, edges can be seen as abrupt changes in intensity, i.e.\ higher values of gradient.
An example of this behavior is shown in Figure \ref{fig:profiles3d}, where gradient was computed using the Sobel operators. 

Then, once the respective operators are applied, an approximation of the gradient (stored in two matrices) 
is obtained, containing the approximation of the partial derivatives $f_x$ and $f_y$. Gradient magnitude 
image $M$ is calculated using the equation \ref{eq:mag_alpha} (Figure \ref{fig:profiles3d-c}). 

Finally, the edges image is obtained by thresholding the gradient magnitude image, i.e.\ for pixel at position $i$ in the image compute
\begin{equation*}
	\mbox{edges}[i,j] = \begin{cases} 0,& \mbox{if}\ M[i,j]\geq\mbox{th} \\
									1,& \mbox{if}\ M[i,j]<\mbox{th}
					  \end{cases}
\end{equation*}approximate
where $\mbox{th}$ is the threshold. A black and white image is obtained, in which edge points are indicated in white (see Figure \ref{fig:thresholding}). 

Note that different thresholds lead to different results. More and thicker edges for small thresholds values, and the opposite for large thresholds values. This threshold, and the others mentioned below, are defined as a percentage of the maximum value 
of the gradient image\footnote{This means no loss of generality in applying the threshold, and makes the threshold adapt to the behavior of different images.}. The advantage of this is to adapt the threshold to the dynamic range of the image. 
Therefore, the parameter $threshold$ in the algorithms takes values ​​between $0$ and $1$

\clearpage

\begin{figure}[t!]
	\centering
	\subfigure[Grayscale image.]{\label{fig:profiles3d-a}\includegraphics[width=0.3\textwidth]{molino_crop_bw.jpg}}
	
	\subfigure[Horizontal derivative $f_x$.]{\label{fig:profiles3d-b}\includegraphics[width=0.3\textwidth]{gradient_image_x.png}}
	\quad
	\subfigure[Vertical derivative $f_y$.]{\label{fig:profiles3d-c}\includegraphics[width=0.3\textwidth]{gradient_image_y.png}}
	\quad
	\subfigure[Gradient module image $M=\sqrt{f_x^2+f_y^2}$.]{\label{fig:profiles3d-d}\includegraphics[width=0.3\textwidth]{gradient_image.png}}
	\caption{Example computation of the gradient. Original image and gradient image computed using Sobel operators (Operators in Figure \ref{fig:sobel}).}
	\label{fig:profiles3d}
\end{figure}

\begin{figure}[t!]
	\centering
	\subfigure[$th = 0.1*\max(M)$.]{\includegraphics[width=0.3\textwidth]{thresholded_image_1.png}}
	\quad
	\subfigure[$th = 0.2*\max(M)$.]{\includegraphics[width=0.3\textwidth]{thresholded_image_2.png}}
	\quad
	\subfigure[$th = 0.4*\max(M)$.]{\includegraphics[width=0.3\textwidth]{thresholded_image_3.png}}
	\caption{Thresholded images. Different values of the threshold applied to the magnitude of the gradient.}
	\label{fig:thresholding}
\end{figure}


%-------------------------------------------------------------------------------

\subsection{Pseudo-code}

The pseudo-code of implemented algorithms is shown in Algorithm \ref{algo:fded}.

\begin{algorithm}[t]
\caption{First derivative edge detection algorithms.}
\label{algo:fded}
\begin{algorithmic}[1]
\REQUIRE input image, threshold $th$.
\STATE $im \leftarrow$ input image
\STATE Define $operator_x$ and $operator_y$ \COMMENT{Roberts, Prewitt or Sobel.}
\STATE $g_x \leftarrow$ convolution($im$,$operator_x$)
\STATE $g_y \leftarrow$ convolution($im$,$operator_y$)
\STATE $max_M \leftarrow 0$
\FORALL{pixel $i$ in image}
	\STATE $M[i] \leftarrow \sqrt{g_x^2+g_y^2}$ \COMMENT{Gradient magnitude.}
	\IF{$M[i]>max_M$}
		\STATE $max_M \leftarrow M[i]$
	\ENDIF
\ENDFOR
\FORALL{pixel $i$ in image}
	\IF{$M[i] \geq th \times max_M$}
		\STATE $im_{OUT}[i] \leftarrow 255$
	\ELSE
		\STATE $im_{OUT}[i] \leftarrow 0$
	\ENDIF
\ENDFOR
\RETURN output image $\leftarrow im_{OUT}$
\end{algorithmic}
\end{algorithm}

%-------------------------------------------------------------------------------

\section{Algorithms based on second derivative}
\label{sec:second}

The edge detection methods discussed in the previous section are simply based on filtering the 
image with different masks, without taking into account the characteristics of the edges or 
noise in the image. 

The two algorithms presented in this section (Marr-Hildreth \cite{AIM-518} [1980] and Haralick \cite{bb20239} [1987]) 
are based on the second derivative of the image, and both take steps to reduce noise before 
detecting edges in the image.

%-------------------------------------------------------------------------------

\subsection{The \textit{Marr} and \textit{Hildreth} algorithm}

The Marr-Hildreth algorithm is a method of detecting edges in digital 
images. It is based on finding zero crossing points of the second derivative
of the image. This can be done in several ways. Two different ways of doing 
this are implemented in this work (see block diagram in Figure 
\ref{fig:blockdiagram2}): convolving the image with a Gaussian kernel and then 
approximating the second derivative (Laplacian) with a 3x3 kernel, or 
convolving the image with a kernel calculated as the Laplacian of a 
Gaussian function. There are more ways to do so, for example, using 
recursive Gaussian filters \cite{Deriche1993Recursively}. 

\begin{figure}[!b]
	\centering
	\includegraphics[width=0.92\textwidth]{blockdiagram2.pdf}
	\caption{Block diagram of Marr-Hildreth algorithm.}
	\label{fig:blockdiagram2}
\end{figure}

The algorithm is divided in two steps, each one described later:
\begin{enumerate}
	\item Convolution of the image with:
	\begin{itemize}
		\item a Laplacian of Gaussian (LoG) kernel, (or)
		\item a Gaussian kernel and then a Laplacian operator.
	\end{itemize}
	\item Search of zero crossing points in the filtered image.
\end{enumerate}

Some auxiliary functions are needed such as Gaussian kernel and Laplacian of a Gaussian 
kernel generation, and 2-D convolution of an image with a given kernel, 
with different boundary conditions. These operations will be discussed 
in detail in section \ref{sec:appendix1}.

%-------------------------------------------------------------------------------

\subsubsection{Gaussian and LoG kernels}

The Marr-Hildreth algorithm consists on convolving the input image $f(x,y)$ with a LoG kernel;
\begin{equation}\label{eq:log}
  g(x,y) = [\nabla^2G(x,y)]\star f(x,y), 
\end{equation}
and then finding the zero crossings of $g(x,y)$ to determine the location of edges in $f(x,y)$. 
Because these are linear processes, equation \ref{eq:log} can be written also as
\begin{equation}
  g(x,y) = \nabla^2[G(x,y)\star f(x,y)]
\end{equation}
indicating that the image can be smoothed with a Gaussian filter first, and then compute the Laplacian of the result\footnote{The difference between these approaches lies in the compromise between the accuracy in the calculation and the computational cost (see Section \ref{kernelcomparison}).} .

% agregar comentario de que la diferencia entre ellos radica en un compromiso entre precision en el calculo
% y costo computacional. Y que se va a explicar mas adelante.

The Marr-Hildreth edge-detection algorithm may be summarized as follows:
\begin{enumerate}
	\item Filter the input image with a $n \times n$ Gaussian lowpass filter obtained by sampling the Gaussian kernel (see equation \ref{eq:gaussian_function}). % reescribir (sampleo de un kernel gaussiano)
	\item Compute the Laplacian of the image resulting from step 1, using, for example, the $3\times3$ mask\footnote{Steps 1 and 2 can be merged into one, using a $n\times n$ LoG lowpass filter obtained by sampling equation \ref{eq:log_function}.}:
	\begin{equation*}
		\begin{bmatrix}
			1 &  1 & 1 \\
			1 & -8 & 1 \\
			1 &  1 & 1 \\
		\end{bmatrix}
	\end{equation*}
	\item Find the zero crossings of the image from step 2.
\end{enumerate}

% Sacar la footnote y explicar de otra manera las dos formas de Marr-Hildreth

%-------------------------------------------------------------------------------

\subsubsection{Zero crossing}

% cambiar la primer frase de este parrafo. Explicar de otra forma lo de buscar que haya al menos un cambio de signo en los pixeles vecinos opuestos.

A zero crossing at pixel $p$ implies that the signs of at least two opposite neighboring pixels are 
different. There are four cases to test: left/right, up/down, and the two diagonals. In this case 
a threshold is used, so that not only the signs of the opposite pixels must differ, also their 
difference in absolute value must be greater than a certain threshold. 

The zero-crossing threshold ($th_{ZC}$) is given as a percentage of the maximum value $max_L$ of the Laplacian 
image (both Gaussian and LoG kernels). Each pixel $p$ has eight neighbors, named according to their position 
as follows.

\begin{center}
\begin{tabular}{ c c c c c }
	$p_{up,left}$		& 					& $p_{up,middle}$	&					& $p_{up,right}$ 		\\
						& $\nwarrow$		& $\uparrow$		& $\nearrow$		&						\\
	$p_{middle,left}$	& $\leftarrow$		& $p$				& $\rightarrow$		& $p_{middle,right}$	\\
						& $\swarrow$		& $\downarrow$		& $\searrow$		&						\\
	$p_{down,left}$		&					& $p_{down,middle}$	&					& $p_{down,right}$.		\\  
\end{tabular}
\end{center}

Then a pixel $p$ is considered as edge pixel if any of the following conditions is true 
(for simplicity the Laplacian image is denoted as $\mathcal{L}$):
\begin{itemize}
	\item $(\mbox{sign}(\mathcal{L}[p_{up,left}])\neq\mbox{sign}(\mathcal{L}[p_{down,right}])$ \& $|\mathcal{L}[p_{up,left}]-\mathcal{L}[p_{down,right}]|>th_{ZC}*max_L$
	\item $(\mbox{sign}(\mathcal{L}[p_{up,middle}])\neq\mbox{sign}(\mathcal{L}[p_{down,middle}])$ \& $|\mathcal{L}[p_{up,middle}]-\mathcal{L}[p_{down,middle}]|>th_{ZC}*max_L$
	\item $(\mbox{sign}(\mathcal{L}[p_{down,left}])\neq\mbox{sign}(\mathcal{L}[p_{up,right}])$ \& $|\mathcal{L}[p_{down,left}]-\mathcal{L}[p_{up,right}]|>th_{ZC}*max_L$
	\item $(\mbox{sign}(\mathcal{L}[p_{middle,left}])\neq\mbox{sign}(\mathcal{L}[p_{middle,right}])$ \& $|\mathcal{L}[p_{middle,left}]-\mathcal{L}[p_{middle,right}]|>th_{ZC}*max_L$. \\
\end{itemize}

Zero crossing detection is the key feature of the Marr-Hildreth edge detection method. The technique 
presented in the previous paragraph is attractive for its simplicity of implementation and its low 
computational cost. In general it is a technique that yields good results, but if more precision in 
finding the zero crossings is needed, more advanced methods for finding zero crossings with subpixel 
accuracy could be used (e.g.\ marching squares \cite{marching_cubes}).

%-------------------------------------------------------------------------------

\subsubsection{Pseudo-code}

The pseudo-code of Marr-Hildreth's algorithm is shown in Algorithm \ref{algo:mh}.

\begin{algorithm}[t!]
\caption{Marr-Hildreth edge detection algorithm.}
\label{algo:mh}
\begin{algorithmic}[1]
\REQUIRE input image, standard deviation $\sigma$, kernel size $n$ and zero-crossing threshold $tzc$.
\STATE $im \leftarrow$ input image
\STATE $kernel \leftarrow$ generate\_kernel($n$,$\sigma$) \COMMENT{Generated Gaussian or LoG kernel}
\STATE $im_{SMOOTHED} \leftarrow$ convolution($im$,$kernel$)
\IF{Gaussian kernel}
	\STATE Define Laplacian operator $laplacian$
	\STATE $im_{LAPL} \leftarrow$ convolution($im$,$laplacian$)
\ELSE
	\STATE $im_{LAPL} \leftarrow im_{SMOOTHED}$
\ENDIF
\STATE $max_L \leftarrow 0$
\FORALL{pixel $i$ in image $im_{LAPL}$}
	\IF{$im_{LAPL}[i]>max_L$}
		\STATE $max_L \leftarrow im_{LAPL}[i]$
	\ENDIF
\ENDFOR
\FORALL{pixel $i$ in image $im_{LAPL}$, except borders}
	\FORALL{pair $(p_1,p_2)$ of opposite neighbors of $p$ in $im_{LAPL}$}
		\IF{($\mbox{sign}(im_{LAPL}[p_1])\neq\mbox{sign}(im_{LAPL}[p_2])$) \AND ($|im_{LAPL}[p_1]-im_{LAPL}[p_2]|>th_{ZC}$)}
			\STATE $im_{OUT}[i] \leftarrow 255$
		\ELSE
			\STATE $im_{OUT}[i] \leftarrow 0$
		\ENDIF
	\ENDFOR
\ENDFOR
\RETURN output image $\leftarrow im_{OUT}$
\end{algorithmic}
\end{algorithm}

%-------------------------------------------------------------------------------

\subsection{The \textit{Haralick} algorithm}

In this section the original work of Haralick \cite{bb20239} on edge detection is presented in detail.
The main idea behind this algorithm is identical to that of the previous method: find zeros in 
the second derivative of the image. In this method, however, the input image is smoothly approximated through local bi-cubic
polynomial fitting. Then, when calculating the second derivative analytically, it is possible to find 
an equivalent expression to find the zeros of the second derivative of the polynomial as a function of 
its parameters.%\footnote{This implementation is slightly different from the traditional implementation 
%of the Haralick algorithm. We do not use the condition concerning the third derivative (to be negative), 
%in order to implement two edge detection methods using only the second derivative (along the 
%Marr-Hildreth algorithm).}

%-------------------------------------------------------------------------------

\subsubsection{Bi-cubic polynomial fitting}
\label{sec:bicubic}

Here, the interpolation method used in the original work of Haralick is presented, although there are other options to do this \cite{getreuer}. 

The surrounding neighborhood of a point $(x,y)$ in the image $f$ is approximated using the following bi-cubic polynomial
\begin{equation}
	\label{eq:bicubic}
	f(x,y) = k_1 + k_2x + k_3y + k_4x^2 + k_5xy + k_6y^2 + k_7x^3 + k_8x^2y + k_9xy^2 + k_{10}y^3. \\
\end{equation}
where $(x,y)$ are the offsets of each neighborhood point relative to the center point (e.g.\ in a neighborhood of size $5\times5$, $x$ and $y$ take values ​​between $-2$ and $2$).

To solve this problem, it is necessary to take more neighbors than coefficients to be adjusted. As there are 10 coefficients to compute, the smallest 
neighborhood of odd size that accomplishes this has size $5\times5$. Having $10$ coefficients and $25$ data 
points leads to an overdetermined system, which can be solved using least squares or any other data fitting technique.
In Haralick's work, this approximation is computed using least squares.

%, and to speed up the calculations the solution is 
%computed by convolving the image with some precomputed masks in Table \ref{table:b_i}, to find the coefficients $k_1\dots k_{10}$ at each point $(x,y)$. \\

Consider 25 points in a small neighborhood of a point $(x,y)$ in the image. This gives us an equal number of equations 
to find 10 coefficients. As mentioned before this leads to an overdetermied system which can be solved by least squares. 
By substituting the 25 data points into the polynomial equation (\ref{eq:bicubic}), the following system of 
equations is obtained,
\begin{equation*}
	\begin{array}{l}
		f_1 = f(x_1,y_1) = k_1 + k_2x_1 + k_3y_1 + k_4x_1^2 + k_5x_1y_1 + k_6y_1^2 + k_7x_1^3 + k_8x_1^2y_1 + k_9x_1y_1^2 + k_{10}y_1^3 \\
		f_2 = f(x_2,y_2) = k_1 + k_2x_2 + k_3y_2 + k_4x_2^2 + k_5x_2y_2 + k_6y_2^2 + k_7x_2^3 + k_8x_2^2y_2 + k_9x_2y_2^2 + k_{10}y_2^3 \\
		\vdots \\
		f_{25} = f(x_{25},y_{25}) = k_1 + k_2x_{25} + k_3y_{25} + k_4x_{25}^2 + k_5x_{25}y_{25} + k_6y_{25}^2 + k_7x_{25}^3 + k_8x_{25}^2y_{25} + k_9x_{25}y_{25}^2 + k_{10}y_{25}^3 . \\
	\end{array}
\end{equation*}

Using matrix notation, the system can be rewritten as

\begin{equation*}
	\begin{bmatrix} 
		f_1		\\ 
		f_2		\\ 
		\vdots	\\
		f_{25}
	\end{bmatrix} 
	= 
	\begin{bmatrix} 
		1 		& x_1 		& y_1 		& x_1^2 	& x_1y_1 		& y_1^2 	& \hdots 	& y_1^3 	\\
		1 		& x_2 		& y_2 		& x_2^2 	& x_2y_2 		& y_2^2 	& \hdots 	& y_2^3 	\\
		\vdots	& \vdots	& \vdots	& \vdots	& \vdots		& \vdots	& \ddots	& \vdots	\\
		1 		& x_{25}	& y_{25}	& x_{25}^2 	& x_{25}y_{25} 	& y_{25}^2 	& \hdots 	& y_{25}^3
	\end{bmatrix}
	\times
	\begin{bmatrix}
		k_1		\\
		k_2		\\
		\vdots	\\
		k_{10}
	\end{bmatrix}
	\Rightarrow \mathbf{f} = \mathbf{A}\mathbf{k} .\\
\end{equation*}

Then, the least squares problem can be approximately solved by computing the pseudo inverse of $\mathbf{A}$,
\begin{equation*}
	(\mathbf{A}^T\mathbf{A})^{-1}\mathbf{A}^T\mathbf{f} = \mathbf{k} \ \ \Rightarrow \ \ \mathbf{k} = \mathbf{B}\mathbf{f} .\\
\end{equation*}
$\mathbf{B}$ is a $10\times25$ matrix, 
\begin{equation*}
	\begin{bmatrix}
		b_{1,1}		& b_{1,2}	& \cdots	& b_{1,25}	\\
		b_{2,1}		& b_{2,2}	& \cdots	& b_{2,25}	\\
		\vdots		& \vdots	& \ddots	& \vdots	\\
		b_{10,1}	& b_{10,2}	& \cdots	& b_{10,25}
	\end{bmatrix}
	\times
	\begin{bmatrix}
		f_1		\\
		f_2		\\
		\vdots	\\
		f_{25}
	\end{bmatrix}
	=
	\begin{bmatrix}
		k_1		\\
		k_2		\\
		\vdots	\\
		k_{10}
	\end{bmatrix} .\\
\end{equation*}
For each coefficient ($i$ from $1$ to $10$), 
\begin{equation}
	\label{eq:coefficients}
	k_i = b_{i,1}f_1 + b_{i,2}f_2 + b_{i,3}f_3 + \cdots + b_{i,25}f_{25}% \ \ \Rightarrow \ \ \mathbf{k_i} = \mathbf{f}\star\mathbf{b_i},\\
\end{equation}
where
\begin{equation}
	\label{eq:b_i}
	\mathbf{b_i} = \begin{bmatrix}	b_{i,1}		& b_{i,2}	& \cdots	& b_{i,5}	\\
									b_{i,6}		& b_{i,7}	& \cdots	& b_{i,10}	\\
									\vdots		& \vdots	& \ddots	& \vdots	\\
									b_{i,21}	& b_{i,22}	& \cdots	& b_{i,25}	\\
					\end{bmatrix}.\\
\end{equation}
%% VER UNA FORMA DE ESCRIBIR TAMBIEN LOS f1...f25 PARA MOSTRAR POR QUE LOS bi,1 ... bi,25 ESTAN DISPUESTOS COMO UNA MATRIZ. ES PORQUE LOS f SON UN ENTORNO 5X5 DEL PIXEL, Y SE MULTIPLIAN PUNTO A PUNTO POR LOS b.

Using equation \ref{eq:coefficients} and the masks given in Table \ref{table:b_i}, it is possible to compute 
the coefficients $k_1 \hdots k_{10}$ for all the points in the image. The elements of the mask $\mathbf{b_i}$ 
are the elements of the i-th row of $\mathbf{B} = (\mathbf{A}^T\mathbf{A})^{-1}\mathbf{A}^T$.

In order to speed up the calculations, the solution is computed by convolving the image with some precomputed 
masks in Table \ref{table:b_i}, to find the coefficients $k_1\dots k_{10}$ at each point $(x,y)$. 

\newcolumntype{C}{>{\centering\arraybackslash}m{1cm}<{}}
\begin{table}
\centering
\subfigure[$\mathbf{b_1}$.] {
\begin{tabular}{|*{5}{C|}}
\hline
425 & 275 & 225 & 275 & 425 \\
\hline
275 & 125 &  75 & 125 & 275 \\ 
\hline
225 &  75 &  25 &  75 & 225 \\
\hline
275 & 125 &  75 & 125 & 275 \\
\hline
425 & 275 & 225 & 275 & 425 \\
\hline
\end{tabular}}
\qquad\qquad
\subfigure[$\mathbf{b_2}$.] {
\begin{tabular}{|*{5}{C|}}
\hline
-2260 & -620 & 0 & 620 & 2260 \\ 
\hline
-1660 & -320 & 0 & 320 & 1660 \\ 
\hline
-1460 & -220 & 0 & 220 & 1460 \\
\hline
-1660 & -320 & 0 & 320 & 1660 \\ 
\hline
-2260 & -620 & 0 & 620 & 2260 \\
\hline
\end{tabular}} \\
\subfigure[$\mathbf{b_3}$.] {
\begin{tabular}{|*{5}{C|}}
\hline
2260  &  1660 &  1460 &  1660 &  2260 \\ 
\hline
620   &   320 &   220 &   320 &   620 \\ 
\hline
0     &     0 &     0 &     0 &     0 \\
\hline
-620  &  -320 &  -220 &  -320 &  -620 \\ 
\hline
-2260 & -1660 & -1460 & -1660 & -2260 \\
\hline
\end{tabular}}
\qquad\qquad
\subfigure[$\mathbf{b_4}$.] {
\begin{tabular}{|*{5}{C|}}
\hline
1130 & 620 & 450 & 620 & 1130 \\ 
\hline
830  & 320 & 150 & 320 &  830 \\ 
\hline
730  & 220 &  50 & 220 &  730 \\
\hline
830  & 320 & 150 & 320 &  830 \\ 
\hline
1130 & 620 & 450 & 620 & 1130 \\
\hline
\end{tabular}} \\
\subfigure[$\mathbf{b_5}$.] {
\begin{tabular}{|*{5}{C|}}
\hline
-400 & -200 & 0 &  200 &  400 \\ 
\hline
-200 & -100 & 0 &  100 &  200 \\ 
\hline
0    &    0 & 0 &    0 &    0 \\
\hline
200  &  100 & 0 & -100 & -200 \\ 
\hline
400  &  200 & 0 & -200 & -400 \\
\hline
\end{tabular}}
\qquad\qquad
\subfigure[$\mathbf{b_6}$.] {
\begin{tabular}{|*{5}{C|}}
\hline
1130 & 830 & 730 & 830 & 1130 \\ 
\hline
620  & 320 & 220 & 320 &  620 \\ 
\hline
450  & 150 &  50 & 150 &  450 \\
\hline
620  & 320 & 220 & 320 &  620 \\ 
\hline
1130 & 830 & 730 & 830 & 1130 \\
\hline
\end{tabular}} \\
\subfigure[$\mathbf{b_7}$.] {
\begin{tabular}{|*{5}{C|}}
\hline
-8260 & -2180 & 0 & 2180 & 8260 \\ 
\hline
-6220 & -1160 & 0 & 1160 & 6220 \\ 
\hline
-5540 &  -820 & 0 &  820 & 5540 \\
\hline
-6220 & -1160 & 0 & 1160 & 6220 \\ 
\hline
-8260 & -2180 & 0 & 2180 & 8260 \\
\hline
\end{tabular}}
\qquad\qquad
\subfigure[$\mathbf{b_8}$.] {
\begin{tabular}{|*{5}{C|}}
\hline
5640  &  3600 &  2920 &  3600 &  5640 \\ 
\hline
1800  &   780 &   440 &   780 &  1800 \\ 
\hline
0     &     0 &     0 &     0 &     0 \\
\hline
-1800 &  -780 &  -440 &  -780 & -1800 \\ 
\hline
-5640 & -3600 & -2920 & -3600 & -5640 \\
\hline
\end{tabular}} \\
\subfigure[$\mathbf{b_9}$.] {
\begin{tabular}{|*{5}{C|}}
\hline
-5640 & -1800 & 0 & 1800 & 5640 \\ 
\hline
-3600 &  -780 & 0 &  780 & 3600 \\ 
\hline
-2920 &  -440 & 0 &  440 & 2920 \\
\hline
-3600 &  -780 & 0 &  780 & 3600 \\ 
\hline
-5640 & -1800 & 0 & 1800 & 5640 \\
\hline
\end{tabular}}
\qquad\qquad
\subfigure[$\mathbf{b_{10}}$.] {
\begin{tabular}{|*{5}{C|}}
\hline
8260  &  6220 &  5540 &  6220 &  8260 \\ 
\hline
2180  &  1160 &   820 &  1160 &  2180 \\ 
\hline
0     &     0 &     0 &     0 &     0 \\
\hline	
-2180 & -1160 &  -820 & -1160 & -2180 \\ 
\hline
-8260 & -6220 & -5540 & -6220 & -8260 \\
\hline
\end{tabular}} \\
\caption{Masks to compute the coefficients of the bicubic fit.}
\label{table:b_i}
\end{table}

%-------------------------------------------------------------------------------

\subsubsection{Analytical calculation of the second derivative}
\label{sec:secderivative}

The neighborhood of each point of the image is approximated using the bi-cubic polynomial 
expression in equation \ref{eq:bicubic}. If just the first order terms of this 
polynomial are taken, the gradient angle $\theta$, defined with negative x-axis, can be approximated as
\begin{align}
	\sin(\theta) & = -\frac{k_2}{\sqrt{k_2^2 + k_3^2}} \nonumber \\
	\cos(\theta) & = -\frac{k_3}{\sqrt{k_2^2 + k_3^2}}.  
\label{eq:sincos}
\end{align}
Now substituting the variables $x$ and $y$ in polar form as
\begin{equation*}
	x = \rho\cos{\theta} \: \text{and} \: y = \rho\sin{\theta} 
\end{equation*}
in the bi-cubic polynomial, we obtain
\begin{equation}
	f_{\theta}(\rho) = C_0 + C_1\rho + C_2\rho^2 + C_3\rho^3 ,
\end{equation}
where
\begin{align}
\label{eq:c}
	C_0 & = k_1 \nonumber \nonumber \\
	C_1 & = k_2\sin(\theta) + k_3\cos(\theta) \nonumber \\
	C_2 & = k_4\sin^2(\theta) + k_5\sin(\theta)\cos(\theta) + k_6\cos^2(\theta) \nonumber \\
	C_3 & = k_7\sin^3(\theta) + k_8\sin^2(\theta)\cos(\theta) + k_9\sin(\theta)\cos^2(\theta) + k_{10}\cos^3(\theta). 
\end{align}
The derivatives are obtained as follows:
\begin{align}
	f'_{\theta}(\rho) = C_1 + 2C_2\rho + 3C_3\rho^2 \nonumber \\
	f''_{\theta}(\rho) = 2C_2 + 6C_3\rho \nonumber \\
	f'''_{\theta}(\rho) = 6C_3 .\nonumber 
\end{align}
Then, the condition that the second derivative is equal to zero becomes
\begin{equation}
	f''_{\theta}(\rho) = 2C_2 + 6C_3\rho = 0 \ \ \Rightarrow \ \ \left| \frac{C_2}{3C_3} \right| < \rho_0,
\end{equation}
where $\rho_0$ is a fixed threshold, passed as an argument to the algorithm. Ideally, the polynomial function 
should be zero at $\rho=0$, but this condition is relaxed, allowing the polynomial to become zero in a neighborhood 
of $\rho=0$ ($\rho\leq\rho_0$). This parameter must be greater than zero and and less than $\sqrt{2}$. Its optimal 
value is between $0.4$ and $0.6$.

The other condition required is that the third derivative to be negative, i.e.\
\begin{equation}
	f'''_{\theta}(\rho) = 6C_3 < 0 \ \ \Rightarrow \ \ C_3 < 0.
\end{equation}
Given the direction defined by $\theta$, with $\rho>0$, an edge will always be an ascending step. This 
indicates that the first derivative in this direction has a maximum and that the second derivative has a zero 
crossing with negative slope. That is, the third derivative less than zero.

%-------------------------------------------------------------------------------

\subsubsection{Algorithm}

The Haralick edge detection algorithm is summarized in the following 4 steps:

\begin{enumerate}
	\item For each pixel in the image, find the coefficients $k_1 \hdots k_{10}$, as shown in section \ref{sec:bicubic}.
	\item Compute $\sin(\theta)$ and $\cos(\theta)$ (equations \ref{eq:sincos}).
	\item Compute $C_2$ and $C_3$ (equations \ref{eq:c}).
	\item If $\left| \frac{C_2}{3C_3} \right| < \rho_0$ and $C_3 < 0$, then that point is an edge point.
\end{enumerate}

%-------------------------------------------------------------------------------

\subsubsection{Pseudo-code}

The pseudo-code of Haralick's algorithm is shown in Algorithm \ref{algo:haralick}.

\begin{algorithm}[t]
\caption{Haralick edge detection algorithm.}
\label{algo:haralick}
\begin{algorithmic}[1]
\REQUIRE input image (width $w$, height $h$), edge point condition threshold $\rho_0$.
\STATE $im \leftarrow$ input image
\STATE Define the ten $5x5$ masks ($mask_1\dots mask_{10}$) used to determine coefficients $k_1$ to $k_{10}$ \COMMENT{See Table \ref{table:b_i}.}
\FORALL{pixel $i$ in image $im$}
	\STATE $neighbors \leftarrow$ $5\times5$ neighborhood of pixel $i$ in image $im$
	\FOR{$j=1$ \TO $10$}
		\STATE $k_j \leftarrow$ $\sum_{n,m}neighbors[n,m]mask_j[n,m]$ % convolution($neighbors$,$mask[j]$)
	\ENDFOR
	\STATE $C_2 \leftarrow \frac{k_2^2k_4 + k_2k_3k_5 + k_3^2k_6}{k_2^2 + k_3^2}$
	\STATE $C_3 \leftarrow \frac{k_2^3k_7 + k_2^2k_3k_8 + k_2k_3^2k_9 + k_3^3k_{10}}{(\sqrt{k_2^2 + k_3^2})^3}$
	\IF{{$|C_2/3C_3|\leq\rho_0$} \AND {$C_3<0$}}
		\STATE $im_{OUT}[i] \leftarrow 255$
	\ELSE
		\STATE $im_{OUT}[i] \leftarrow 0$
	\ENDIF
\ENDFOR
\RETURN output image $\leftarrow im_{OUT}$
\end{algorithmic}
\end{algorithm}

%-------------------------------------------------------------------------------

\section{Common Mathematical Operations}
\label{sec:appendix1}

All algorithms implemented use some common mathematical operations that are independent of the algorithms themselves. 
These operations, although basic, have a great impact on the algorithm outcome, and thus they need to be 
implemented with care. In this section those operations are reviewed.

%-------------------------------------------------------------------------------

%\subsection{Grayscale conversion}
%\label{sec:grayscale}

%The first step in every algorithm presented above is to convert color images to gray intensity images. 
%For this purpose, \href{http://www.libpng.org/}{libpng} coefficients are used,
%\begin{equation}
%    Y = (6968 R + 23434 G + 2366 B) / 32768
%\end{equation}
%where $R$, $G$ and $B$ are the red, green and blue components respectively\footnote{Must be carefully implemented, using \texttt{float} to avoid truncated results. See code in section \ref{app:marr-hildreth}.}.

%-------------------------------------------------------------------------------

\subsection{Kernel generation}

Some of the algorithms presented above require the use of a Gaussian kernel or a LoG kernel. These 
kernels are generated by sampling the corresponding analytical function, which in each case depends 
on the standard deviation $\sigma$ of the Gaussian function. The result is an array of 
size $n$ by $n$.

% Explicar que voy a crear un patch centrado en [x,y]=[0,0], y un poco mas de la generacion del patch.
% Explicar que se samplea en el centro de los pixeles (hay varias formas de generar el patch, por ejemplo,
% que cada valor sea la media de la guncion en el pixel, etc).

%-------------------------------------------------------------------------------

\subsubsection{Gaussian kernel}

Gaussian kernel is generated by sampling the 2-D Gaussian function (centered at $(0,0)$)
%\footnote{Note that for simplicity the normalizing coefficient $1/\sqrt{2\pi\sigma^2}$ is omitted.}
\begin{equation}
	\label{eq:gaussian_function}
	G(x,y) = \frac{1}{\sqrt{2\pi\sigma^2}}e^{-\frac{x^2+y^2}{2\sigma^2}}
\end{equation}
where $\sigma$ is the standard deviation (sometimes $\sigma$ is called the \textit{scale space constant}).

The size of the kernel $n$ and the standard deviation of the exponential function $\sigma$ are both 
input parameters, but these are not strictly independent of each other. 
A value of $n$ large enough is needed to ensure that no information is lost when creating the kernel $G$. To ensure this, $n$ is taken equal to the first odd integer greater than $6\sigma$. Larger values of $n$ do not add more significant samples ​​of $G$, and increase the number of operations in the convolution.

Figure \ref{fig:gaussian_kernel} shows a Gaussian kernel, generated with $\sigma = 4$ and $n = 25$ 
(first odd integer greater than $6\sigma=24$).

\begin{SCfigure}[][!t]
	\centering
	\includegraphics[width=0.5\textwidth]{kernel_gaussian.pdf}
	\caption{Gaussian kernel, $\sigma=4$, $n=25$. Is easy to see that the selected value of n is 
large enough to have a good approximation of the Gaussian function in the kernel.}
	\label{fig:gaussian_kernel}
\end{SCfigure}

%-------------------------------------------------------------------------------

\subsubsection{LoG kernel}

The Laplacian of Gaussian $\nabla^2G(x,y)$ can be obtained analyically first and then a discrete mask 
can be computed by sampling the analytical kernel. Using this kernel for edge detection involves only 
one convolution with the input image (unlike the case of Gaussian kernel, in which two convolutions 
have to be performed, one with the kernel and another with the Laplacian operator), but the kernel 
support must be greater in order to obtain the same precision..

% Una linea explicando que para la misma presicion se requiere un soporte mas grande (y linkar a la seccion siguiente).

\begin{equation}
	LoG \stackrel{\triangle}{=}\nabla^2G(x,y)=\frac{\partial^2}{\partial^2 x}G(x,y) + \frac{\partial^2}{\partial^2 y}G(x,y)
\end{equation}

We first compute

\begin{equation} 
	\frac{\partial}{\partial x}G(x,y)=-\frac{1}{\sqrt{2\pi\sigma^2}}\frac{x}{\sigma^2}e^{-(x^2+y^2)/2\sigma^2},
\end{equation}
\begin{equation} 
	\frac{\partial^2}{\partial^2 x}G(x,y)=\frac{1}{\sqrt{2\pi\sigma^2}}\frac{x^2-\sigma^2}{\sigma^4}e^{-(x^2+y^2)/2\sigma^2},
\end{equation}
\begin{equation} 
	\frac{\partial^2}{\partial^2 y}G(x,y)=\frac{1}{\sqrt{2\pi\sigma^2}}\frac{y^2-\sigma^2}{\sigma^4}e^{-(x^2+y^2)/2\sigma^2},
\end{equation}
Therefore we obtain
\begin{equation}
	\label{eq:log_function}
	LoG(x,y)=\frac{1}{\sqrt{2\pi\sigma^2}}\frac{x^2+y^2-2\sigma^2}{\sigma^4}e^{-(x^2+y^2)/2\sigma^2}.
\end{equation}\\

Now the LoG kernel is generated by sampling the function defined in equation \ref{eq:log_function}. 
Figure \ref{fig:log_kernel} shows a LoG kernel, generated using the values $\sigma=4$ and $n=31$.

\begin{SCfigure}[][!t]
	\centering
	\includegraphics[width=0.5\textwidth]{kernel_log.pdf}
	\caption{Laplacian of a Gaussian kernel, $\sigma=4$, $n=31$. The selected value of n is sufficient 
to have a good approximation of the LoG function in the kernel, but is greater than in the case of 
Gaussian kernel.}
	\label{fig:log_kernel}
\end{SCfigure}

%-------------------------------------------------------------------------------

\subsubsection{Gaussian and LoG functions comparison}
\label{kernelcomparison}

As mentioned before, the size $n$ of the kernel and the standard deviation $\sigma$ of the exponential function 
are not independent of each one. This is because the function LoG has wider support than the
Gaussian function (i.e. LoG function has a slower decay), so a greater value of $n$ is needed to 
generate a correctly sampled LoG kernel than in the case of the Gaussian kernel, with the same $\sigma$.\\

Figure \ref{fig:kernels} shows both functions generated with the same value of $\sigma$. Is clearly 
required a larger kernel size in the case of the LoG function (approximately 18\% more). For 
example, using $\sigma=4$, the optimum value for $n$ in the case of the Gaussian function is the 
first odd integer greater than $6\sigma$, which is $25$, and for the case of LoG function, would 
be $29$.
% Especificar cual es el criterio para el tamaño del LoG.
% mirar en el software si no es mejor quitar la dependencia con n, y que se decida internamente el tamaño
% del kernel en funcion de sigma.

\begin{SCfigure}[][!t]
	\centering
	\includegraphics[width=0.5\textwidth]{kernels.pdf}
	\caption{Comparison of the Gaussian and LoG functions.}
	\label{fig:kernels}
\end{SCfigure}

%-------------------------------------------------------------------------------

\subsection{Convolution}

When making a convolution, it is necessary to define the boundary conditions used to calculate such convolution 
around the edges of the image, and to get a valid output the same size as input image. In this paper, 
two methods were implemented: zero-padding and boundary reflection. Zero-padding implies complete 
the borders of the image with many zero valued pixels as needed (depending on the size of the convolution 
kernel). Reflection involves completing those pixels with the corresponding symmetric pixel value 
relative to the edge of the image. Direct convolution is not the only way of filtering an image with a kernel. There are other methods such as FFT convolution or recursive filtering\footnote{Implementations of these methods can be found in the \href{https://tools.ipol.im/wiki/author/code/hatchery/}{IPOL Code Hatchery}.}.

%-------------------------------------------------------------------------------

\begin{comment}

\section{Implementation}
\label{sec:appendix2}

In this section a detailed explanation of the source code of the different implemented algorithms
is presented. \\

To the best of our knowledge, the only freely available codes implementing these algorithms are the Matlab implementation of edge detection algorithms using 2D masks (\href{http://www.mathworks.com/help/toolbox/images/ref/edge.html}{\texttt{edge} command}) and a Haralick algorithm implementation in Matlab published in \href{http://www.mathworks.com/matlabcentral/fileexchange/35329-simple-edge-detection-using-classical-haralick-method}{Matlab Central}. There is also an implementation similar to the Marr-Hildreth algorithm, as well available in \href{http://www.mathworks.com/matlabcentral/fileexchange/11572-sdgd-edge-detection-filter}{Matlab Central}. All these links were last checked in July 2012. \\

%This code documentation was generated using a perl script to extract some parts of the 
%code and comments into latex files. This code documentation tool is available as a \href{https://github.com/juan-cardelino/source_comment_extractor}{GIT project}. \\

%Also a Doxygen generated documentation of the functions used is available \href{http://iie.fing.edu.uy/~haldos/ipol/red_v0.1}{here}. \\

Required external libraries: 
\begin{itemize}
	\item \href{http://dev.ipol.im/git/coco/iio.git}{\texttt{iio}}.
	\item \href{http://www.libpng.org/}{\texttt{libpng}}. 
\end{itemize}

\subsection{Roberts, Prewitt and Sobel}
% Please do NOT edit this file.
% It has been automatically generated
% by a perl script from the original cxx sources
% in the Insight/Examples directory

% Any changes should be made in the file
% src/test_fded.c


The source code for this section can be found in the file \verb|test_fded.c|.\\  First derivative edge detectors (Roberts, Prewitt and Sobel), main C file.\\
  
  Parameters:
	\begin{itemize}
		\item \texttt{input\_image} - Input image.
		\item \texttt{threshold} - Threshold for gradient image ($0\leq th \leq 1$).
		\item \texttt{padding\_method} - Padding method flag (in convolution): 0 means zero-padding, 1 means image boundary reflection.
	\end{itemize}

	\textit{Note: Output images are saved with the filenames ``\texttt{roberts.png}'', ``\texttt{prewitt.png}'' and ``\texttt{sobel.png}''.} \\

	Includes:
\small
\begin{lstlisting}
	#include "iio.c"
	#include "2dconvolution.c"
	#include <time.h>
\end{lstlisting}
\vspace{1ex}
\normalsize
  Macros:
\small
\begin{lstlisting}
#define MAX(x, y) (((x) > (y)) ? (x) : (y))
#define MIN(x, y) (((x) < (y)) ? (x) : (y))
#define THRESHOLD(x, th) (((x) > (th)) ? (255) : (0))
\end{lstlisting}
\vspace{1ex}
\normalsize
	\vspace{0.5cm}
	\Large{Main function} \\
\small
\begin{lstlisting}
int main(int argc, char *argv[]) {
\end{lstlisting}
\vspace{1ex}
\normalsize
	Load input image (using \textit{iio}): \\
\small
\begin{lstlisting}
		int w, h, pixeldim;
		float *im_orig = iio_read_image_float_vec(argv[1], &w, &h, &pixeldim);
\end{lstlisting}
\vspace{1ex}
\normalsize
	Grayscale conversion (if necessary): explained in \ref{sec:grayscale}. \\ \\
	Define the normalized Roberts, Prewitt and Sobel operators: \\
	(We use $3\times 3$ operators)
	\begin{itemize}
		\item	Roberts:
				$$
				R_1 = \begin{bmatrix} -1 & 0 & 0 \\ 0 & 1 & 0 \\ 0 & 0 & 0 \end{bmatrix}
				$$
				$$
				R_2 = \begin{bmatrix} 0 & -1 & 0 \\ 1 & 0 & 0 \\ 0 & 0 & 0 \end{bmatrix}
				$$
		\item	Prewitt:
				$$
				P_1 = \begin{bmatrix} \frac{-1}{6} & \frac{-1}{6} & \frac{-1}{6} \\ 0 & 0 & 0 \\ \frac{1}{6} & \frac{1}{6} & \frac{1}{6} \end{bmatrix}
				$$
				$$
				P_2 = \begin{bmatrix} \frac{-1}{6} & 0 & \frac{1}{6} \\ \frac{-1}{6} & 0 & \frac{1}{6} \\ \frac{-1}{6} & 0 & \frac{1}{6} \end{bmatrix}
				$$
		\item	Sobel:
				$$
				S_1 = \begin{bmatrix} \frac{-1}{8} & \frac{-1}{4} & \frac{-1}{8} \\ 0 & 0 & 0 \\ \frac{1}{8} & \frac{1}{4} & \frac{1}{8} \end{bmatrix}
				$$
				$$
				S_2 = \begin{bmatrix} \frac{-1}{8} & 0 & \frac{1}{8} \\ \frac{-1}{4} & 0 & \frac{1}{4} \\ \frac{-1}{8} & 0 & \frac{1}{8} \end{bmatrix}
				$$
	\end{itemize}
\small
\begin{lstlisting}
		double roberts_1[9] = {-1, 0, 0, 0, 1, 0, 0, 0, 0};		// ROBERTS
		double roberts_2[9] = { 0,-1, 0, 1, 0, 0, 0, 0, 0};		// OPERATORS
		//---------------------------------------------------------------------------------
		double prewitt_1[9] = {-1,-1,-1, 0, 0, 0, 1, 1, 1};		// PREWITT
		double prewitt_2[9] = {-1, 0, 1,-1, 0, 1,-1, 0, 1};		// OPERATORS
		//---------------------------------------------------------------------------------
		double sobel_1[9] = {-1,-2,-1, 0, 0, 0, 1, 2, 1};		// SOBEL
		double sobel_2[9] = {-1, 0, 1,-2, 0, 2,-1, 0, 1};		// OPERATORS
		//---------------------------------------------------------------------------------
		for (z=0;z<9;z++) {										// NORMALIZATION
			roberts_1[z] /= sqrt(2);
			roberts_2[z] /= sqrt(2);
			prewitt_1[z] /= sqrt(6);
			prewitt_2[z] /= sqrt(6);
			sobel_1[z] /= sqrt(12);
			sobel_2[z] /= sqrt(12);
		}
\end{lstlisting}
\vspace{1ex}
\normalsize
	The input image is convolved with the defined operatos, using the \texttt{conv2d} function in \texttt{2dconvolution.c}:
\small
\begin{lstlisting}
		int padding_method = atoi(argv[3]);
		double *im_r1 = conv2d(im, w, h, roberts_1, 3, padding_method);
		double *im_r2 = conv2d(im, w, h, roberts_2, 3, padding_method);
		double *im_p1 = conv2d(im, w, h, prewitt_1, 3, padding_method);
		double *im_p2 = conv2d(im, w, h, prewitt_2, 3, padding_method);
		double *im_s1 = conv2d(im, w, h, sobel_1, 3, padding_method);
		double *im_s2 = conv2d(im, w, h, sobel_2, 3, padding_method);
\end{lstlisting}
\vspace{1ex}
\normalsize
	Allocate memory for final images:
\small
\begin{lstlisting}
		float *im_roberts = malloc(w*h*sizeof(float));
		float *im_prewitt = malloc(w*h*sizeof(float));
		float *im_sobel = malloc(w*h*sizeof(float));
\end{lstlisting}
\vspace{1ex}
\normalsize
	For each method, two images are obtained (one for each operator). Then the gradient magnitude image is constructed using $M=\sqrt{g_x^2+g_y^2}$. \\
	Also the absolute maximum value of the constructed images is computed, for each method. \\
\small
\begin{lstlisting}
		int i,j, fila, col;
		double max_r = 0;
		double max_p = 0;
		double max_s = 0;
		int imax = w*h;
		for (i=0;i<imax;i++){
			fila = (int)(i/w);
			col = i - w*fila + 1;
			fila += 1;
			j = col + (w+2)*fila;
			// Max in each case
			im_roberts[i] = sqrt(im_r1[j]*im_r1[j] + im_r2[j]*im_r2[j]);
			im_prewitt[i] = sqrt(im_p1[j]*im_p1[j] + im_p2[j]*im_p2[j]);
			im_sobel[i] = sqrt(im_s1[j]*im_s1[j] + im_s2[j]*im_s2[j]);
			// Absolute max
			max_r = MAX(max_r,im_roberts[i]);
			max_p = MAX(max_p,im_prewitt[i]);
			max_s = MAX(max_s,im_sobel[i]);
		}
\end{lstlisting}
\vspace{1ex}
\normalsize
	Thresholded images of each method are created, using the THRESHOLD macro: \\
\small
\begin{lstlisting}
		float th = atof(argv[2]);
		for (i=0;i<imax;i++){
			im_roberts[i] = THRESHOLD(im_roberts[i],th*max_r);
			im_prewitt[i] = THRESHOLD(im_prewitt[i],th*max_p);
			im_sobel[i] = THRESHOLD(im_sobel[i],th*max_s);
		}
\end{lstlisting}
\vspace{1ex}
\normalsize
	Save output image (using \textit{iio}): \\
\small
\begin{lstlisting}
		iio_save_image_float_vec("roberts.png", im_roberts, w, h, 1);
		iio_save_image_float_vec("prewitt.png", im_prewitt, w, h, 1);
		iio_save_image_float_vec("sobel.png", im_sobel, w, h, 1);
\end{lstlisting}
\vspace{1ex}
\normalsize


\subsection{Marr-Hildreth}
\label{app:marr-hildreth}
% Please do NOT edit this file.
% It has been automatically generated
% by a perl script from the original cxx sources
% in the Insight/Examples directory

% Any changes should be made in the file
% src/test_mh.c


The source code for this section can be found in the file \verb|test_mh.c|.\\  Marr-Hildreth edge detector, main C file.\\
  
  Parameters:
	\begin{itemize}
		\item \texttt{input\_image} - Input image.
		\item \texttt{sigma} - Standard deviation $\sigma$ of the Gaussian kernel.
		\item \texttt{n} - Size $n$ of the Gaussian kernel ($n\times n$).
		\item \texttt{tzc} - Threshold in the Zero-Crossing calculation ($0\leq t_{zc}\leq 1$).
		\item \texttt{padding\_method} - Padding method flag (in convolution): 0 means zero-padding, 1 means image boundary reflection.
		\item \texttt{output\_image} - Output image (edges).
	\end{itemize}

	Includes:
\small
\begin{lstlisting}
	#include "iio.c"
	#include "gaussian_kernel.c"
	#include "2dconvolution.c"
	#include <time.h>
\end{lstlisting}
\vspace{1ex}
\normalsize
	\vspace{0.5cm}
	\Large{Main function} \\
\small
\begin{lstlisting}
int main(int argc, char *argv[]) {
\end{lstlisting}
\vspace{1ex}
\normalsize
	Load input image (using \textit{iio}): \\
\small
\begin{lstlisting}
		int w, h, pixeldim;
		float *im_orig = iio_read_image_float_vec(argv[1], &w, &h, &pixeldim);
\end{lstlisting}
\vspace{1ex}
\normalsize
	Grayscale conversion (if necessary): \\
	First is allocated the memory for the grayscale image \texttt{im}, with
	the corresponding correct allocation check. Then the number of channels of the image is checked: if
	\texttt{pixeldim}=3, RGB image is assumed and conversion is needed, else, single channel image (grayscale) is assumed and
	no conversion is required.\\
	The computation of the gray intensity from RGB levels is:
	$$
	I = \frac{6968\times (\text{float})R + 23434\times (\text{float})G + 2366\times (\text{float})B}{32768} \\
	$$
  To perform this operation, a cast is made to float on the values ​​$R$, $G$ and $B$ of the original image. This can be slow, 
  but ensures the correct image conversion. \\
  These coefficients also ensure there is no saturation in the calculation, since $\frac{6968\times 255 + 23434\times 255 + 2366\times 255}{32768} = \frac{8355840}{32768} = 255$.
\small
\begin{lstlisting}
		double *im = malloc(w*h*sizeof(double));
		if (im == NULL){
			fprintf(stderr, "Out of memory...\n");
			exit(EXIT_FAILURE);
		}
		int z;
		int zmax = w*h;	
		if (pixeldim==3){
			for(z=0;z<zmax;z++){
				im[z] =  (double)(6968*im_orig[3*z] + 23434*im_orig[3*z + 1] 
													+ 2366*im_orig[3*z + 2])/32768;
			}
			fprintf(stderr, "images converted to grayscale\n");
		} else {
			for(z=0;z<zmax;z++){
				im[z] = (double)im_orig[z];
			}
			fprintf(stderr, "images are already in grayscale\n");
		}
\end{lstlisting}
\vspace{1ex}
\normalsize
	Generate Gaussian kernel using the \texttt{gaussian\_kernel} function in \texttt{gaussian\_kernel.c}: \\
\small
\begin{lstlisting}
		double *kernel = gaussian_kernel(n,sigma);
\end{lstlisting}
\vspace{1ex}
\normalsize
	Smooth input image with the Gaussian kernel previously generated, using the \texttt{conv2d} function in \texttt{2dconvolution.c}: \\
\small
\begin{lstlisting}
		double *im_smoothed = conv2d(im, w, h, kernel, n, padding_method);
\end{lstlisting}
\vspace{1ex}
\normalsize
	Computation of the Laplacian of the smoothed image: \\
	A $3\times 3$ approximation of the laplacian operator is used: \\
	$$
	\begin{bmatrix}
		1 &  1 & 1 \\
		1 & -8 & 1 \\
		1 &  1 & 1 
	\end{bmatrix}
	$$
	Using the \texttt{conv2d} function, the \texttt{laplacian} image is obtained. \\
\small
\begin{lstlisting}
		double operator[9] = {1, 1, 1, 1, -8, 1, 1, 1, 1};
		double *laplacian = conv2d(im_smoothed, w+n-1, h+n-1, operator, 3, padding_method);
\end{lstlisting}
\vspace{1ex}
\normalsize
	Now the maximum absolute value of the \texttt{laplacian} image is calculated. This value is required
	for thresholding in zero-crossing calculation. \\
\small
\begin{lstlisting}
		double max_l = 0;
		int p;
		int pmax = (w+n+1)*(h+n+1);
		for (p=0;p<pmax;p++){
			if (abs(laplacian[p])>max_l){
				max_l = abs(laplacian[p]);
			}
		}
\end{lstlisting}
\vspace{1ex}
\normalsize
	Zero-crossing: \\
	The image is explored, looking in every pixel a change of sign between neighboring opposite pixels.
 	In every pixel $p$ the funcion \texttt{get\_neighborhood} (in \texttt{2dconvolution.c}) is used to get the
	9 pixels of its neighborhood:
	$$
	\begin{bmatrix}
		p_{up,left}		& p_{up,middle}		& p_{up,right}		\\
		p_{middle,left}	& p					& p_{middle,right}	\\
		p_{down,left}	& p_{down,middle}	& p_{down,right}	
	\end{bmatrix}
	$$
	Then the pixel $p$ is marked as edge pixel if it occurs that:
	\begin{itemize}
	\item $sign(p_{up,left}) \neq sign(p_{down,right})$, or 
	\item $sign(p_{up,middle}) \neq sign(p_{down,middle})$, or 
	\item $sign(p_{up,right}) \neq sign(p_{down,left})$, or
	\item $sign(p_{middle,left}) \neq sign(p_{middle,right})$. \\
	\end{itemize}
\small
\begin{lstlisting}
		float *zero_cross = calloc(w*h,sizeof(float));		
						// this image will only content values 0 and 255
						// but float type is used for saving using iio.
		if (zero_cross == NULL){
			fprintf(stderr, "Out of memory...\n");
			exit(EXIT_FAILURE);
		}
		int ind_en_lapl, fila, col;
		int *offsets = get_neighbors_offset(w+n+1, 3);
		pmax = w*h;
		int dif_fila_col = (n+1)/2;
		for (p=0;p<pmax;p++){
			fila = ((int)(p/w));
			col = p-(w*fila) + dif_fila_col;
			fila += dif_fila_col;
			ind_en_lapl = col + (w+n+1)*fila;
			double *n3 = get_neighborhood(laplacian, ind_en_lapl, 3, offsets);
			if ((n3[3]*n3[5]<0)&&(abs(n3[3]-n3[5])>(tzc*max_l))) {
				// horizontal sign change
				zero_cross[p] = 255;
			} else if ((n3[1]*n3[7]<0)&&(abs(n3[1]-n3[7])>(tzc*max_l))) {
					// vertical sign change
					zero_cross[p] = 255;
				} else if ((n3[2]*n3[6]<0)&&(abs(n3[2]-n3[6])>(tzc*max_l))) {
						// +45deg sign change
						zero_cross[p] = 255;
					} else if ((n3[0]*n3[8]<0)&&(abs(n3[0]-n3[8])>(tzc*max_l))) {
							// -45deg sign change
							zero_cross[p] = 255;
						}
			free_neighborhood(n3);
		}
		free_neighbors_offsets(offsets);
\end{lstlisting}
\vspace{1ex}
\normalsize
	Save output image (using \textit{iio}): \\
\small
\begin{lstlisting}
		iio_save_image_float_vec(argv[6], zero_cross, w, h, 1);
\end{lstlisting}
\vspace{1ex}
\normalsize
	\textit{Note: the main function in \texttt{test\_mh\_log.c} is essentially the same. The only difference is that a LoG kernel is generated (instead of a Gaussian kernel) using the
	\texttt{LoG\_kernel} function in \texttt{gaussian\_kernel.c}. Therefore there is no need to use the laplacian operator, so only one convolution is made.} \\


\subsection{Haralick}
% Please do NOT edit this file.
% It has been automatically generated
% by a perl script from the original cxx sources
% in the Insight/Examples directory

% Any changes should be made in the file
% src/test_haralick.c


The source code for this section can be found in the file \verb|test_haralick.c|.\\  Haralick edge detectors, main C file.\\
  
  Parameters:
	\begin{itemize}
		\item \texttt{input\_image} - Input image.
		\item \texttt{rhozero} - Threshold for the Haralick condition $|\frac{C_2}{2C_3}|\leq\rho_0$.
		\item \texttt{padding\_method} - Padding method flag (in convolution): 0 means zero-padding, 1 means image boundary reflection.
		\item \texttt{output\_image} - Output image (edges).
	\end{itemize}

	Includes:
\small
\begin{lstlisting}
	#include "iio.c"
	#include "2dconvolution.c"
	#include <time.h>
\end{lstlisting}
\vspace{1ex}
\normalsize
	\vspace{0.5cm}
	\Large{Main function} \\
\small
\begin{lstlisting}
int main(int argc, char *argv[]) {
\end{lstlisting}
\vspace{1ex}
\normalsize
	Load input image (using \textit{iio}): \\
\small
\begin{lstlisting}
		int w, h, pixeldim;
		float *im_orig = iio_read_image_float_vec(argv[1], &w, &h, &pixeldim);
\end{lstlisting}
\vspace{1ex}
\normalsize
	Grayscale conversion (if necessary): explained in \ref{app:marr-hildreth}. \\ \\
	Masks calculated by 2-d fitting (using LS) with the function: 
	$$
	f(x,y) = k_1 + k_2x + k_3y + k_4x^2 + k_5xy + k_6*y^2 + k_7x^3 + k_8x^2y + k_9xy^2 + k_{10}y^3 \\
	$$
\small
\begin{lstlisting}
		double masks[10][25] = { {   425,   275,  225,  275,  425,   
									 275,   125,   75,  125,  275,   
                                     225,    75,   25,   75,  225,   
									 275,   125,   75,  125,  275,   
									 425,   275,  225,  275,  425},
								 { -2260,  -620,    0,  620,  2260, 
								   -1660,  -320,    0,  320,  1660, 
								   -1460,  -220,    0,  220,  1460,
                                   -1660,  -320,    0,  320,  1660, 
                                   -2260,  -620,    0,  620, 2260},
								// matrix continues, too large to display in documentation.
\end{lstlisting}
\vspace{1ex}
\normalsize
	Initialise edge image using \texttt{calloc}:
\small
\begin{lstlisting}
		float *edges = calloc(w*h,sizeof(float));
\end{lstlisting}
\vspace{1ex}
\normalsize
	Padding: a larger auxiliar image \texttt{aux} is required to compute the coefficients $k_1$ to $k_{10}$ in every pixel of the original image. \\
	Two different methods are implemented: zero-padding (\texttt{padding\_method}$=0$) and reflection of original image (\texttt{padding\_method}$=1$).
\small
\begin{lstlisting}
		int wx = (w+8);
		int hx = (h+8);
		double *aux = calloc(wx*hx,sizeof(double));
		int fila,col;
		int imax = wx*hx;
		if (padding_method == 0) {
			for(i=0;i<imax;i++){
				fila = (int)(i/wx);
				col = i-(wx*fila);	
				if ( (fila>=4)&&(col>=4)&&(fila<h+4)&&(col<w+4) ) {
					aux[i] = im[(col-4)+(w*(fila-4))];
				}
			}
		}
\end{lstlisting}
\vspace{1ex}
\normalsize
\small
\begin{lstlisting}
		if (padding_method == 1) {
			int fila_refl, col_refl;
			for(i=0;i<imax;i++){
				fila = (int)(i/wx);
				col = i-(wx*fila);
				if (fila<4) {
					fila_refl = 7 - fila;
					if (col<4) { //zone1
						col_refl = 7 - col;
					} else if (col<w+4) {	//zone2
						col_refl = col;
					} else { //zone3
						col_refl = 2*w + 7 - col;
					}
				} else if (fila<h+4) {
					fila_refl = fila;
					if (col<4) { //zone4
						col_refl = 7 - col;
					} else if (col<w+4) { //image
						col_refl = col;
					} else { //zone5
						col_refl =  2*w + 7 - col;
					}
				} else {
					fila_refl = 2*h + 7 - fila;
					if (col<4) { //zone6
						col_refl =	7 - col;
					} else if (col<w+4) {	//zone7
						col_refl = col;
					} else { //zone8
						col_refl =  2*w + 7 - col;
					}
				}
				aux[i] = im[(col_refl-4)+(w*(fila_refl-4))];
			} //for
		}
\end{lstlisting}
\vspace{1ex}
\normalsize
	Haralick algorithm: coefficients $k_1$ to $k_{10}$ are computed in every pixel of the original image 
	(using the function \texttt{get\_neighbors\_offset} to get the index offsets of the neighbor pixels and 
	the function \texttt{get\_neighborhood} to get the neighborhood of a pixel using those index offsets). 
	Once the coefficients are calculated, are computed
	$$
	C_2 = \frac{k_2^2k_4 + k_2k_3k_5 + k_3^2k_6}{k_2^2 + k_3^2}
	$$
	and
	$$
	C_3 = \frac{k_2^3k_7 + k_2^2k_3k_8 + k_2k_3^2k_9 + k_3^3k_{10}}{(\sqrt{k_2^2 + k_3^2})^3},
	$$
	and then the edge condition is evaluated in every pixel; $|\frac{C_2}{2C_3}|\leq\rho_0$. \\
\small
\begin{lstlisting}
		int i_zp, u, v, num_edges;
		num_edges = 0;
		double k[10];
		int *offsets = get_neighbors_offset(wx, 5);
		double acum;
		double C2, C3, denom, sintheta, costheta;
		for(fila=0;fila<h;fila++){
			for(col=0;col<w;col++){
				i = col + w*fila;				// original image & edges image index
				i_zp = (col+4) + wx*(fila+4);	// padded image index
				double *neighborhood = get_neighborhood(aux, i_zp, 5, offsets);
				// k1 to k10 (note: k1 (u=0) is not necessary)
				for(u=0;u<10;u++){
					acum = 0;
					for(v=0;v<25;v++){
						acum += neighborhood[v]*masks[u][v];
					}
					k[u] = acum;
				}
				// compute C2 and C3
				denom = sqrt( k[1]*k[1] + k[2]*k[2] );
				sintheta = - k[1] / denom;
				costheta = - k[2] / denom;
				C2 = k[3]*sintheta*sintheta + k[4]*sintheta*costheta + k[5]*costheta*costheta;
				C3 = k[6]*sintheta*sintheta*sintheta + k[7]*sintheta*sintheta*costheta +
					 k[8]*sintheta*costheta*costheta + k[9]*costheta*costheta*costheta;
				//if ((fabs(C2 / (3*C3))<=rhozero)&&(C3<=0)) {
				if ((fabs(C2 / (3*C3))<=rhozero)) {
					edges[i] = 255;
					num_edges += 1;
				}
				// free neighborhood
				free_neighborhood(neighborhood);
			}
		}
\end{lstlisting}
\vspace{1ex}
\normalsize
	Save output image (using \textit{iio}): \\
\small
\begin{lstlisting}
		iio_save_image_float_vec(argv[4], edges, w, h, 1);
\end{lstlisting}
\vspace{1ex}
\normalsize


\subsection{Gaussian kernel generation}
% Please do NOT edit this file.
% It has been automatically generated
% by a perl script from the original cxx sources
% in the Insight/Examples directory

% Any changes should be made in the file
% src/gaussian_kernel.c


The source code for this section can be found in the file \verb|gaussian_kernel.c|.\\	This file implements the necessary functions for generating 
	Gaussian and LoG (Laplacian of a Gaussian) kernels. It is also 
	done here the alloc and free of the memory needed. \\

	Includes:
\small
\begin{lstlisting}
	#include <math.h> // exp
	#include <stdlib.h> // malloc, calloc, free
	#include <stdio.h> // fprintf
\end{lstlisting}
\vspace{1ex}
\normalsize
	\vspace{0.5cm}
	\Large{Function \texttt{gaussian\_kernel}} \\
 
	\normalsize
	This function generates a Gaussian kernel of size $n\times n$ and standard deviation $\sigma$. \\
\small
\begin{lstlisting}
double *gaussian_kernel(int n, float sigma){
\end{lstlisting}
\vspace{1ex}
\normalsize
	Memory allocation for the kernel: \\
\small
\begin{lstlisting}
	double *kernel = malloc(n*n*sizeof(double));
\end{lstlisting}
\vspace{1ex}
\normalsize
	A normalized Gaussian kernel is generated using the expression $e^{-\frac{x^2+y^2}{2\sigma^2}}$. \\
\small
\begin{lstlisting}
	int i;
	int fila, col, x, y;
	double suma = 0;
	int imax = n*n;
	for(i=0;i<imax;i++){
		fila = (int)(i/n);
		col = i-(n*fila);
		y = ((int)(n/2))-fila;
		x = col-((int)(n/2));
		kernel[i] = exp(-(x*x + y*y)/(2*sigma*sigma));
		suma += kernel[i];
	}
\end{lstlisting}
\vspace{1ex}
\normalsize
	Kernel normalization, using the sum of its components. \\
\small
\begin{lstlisting}
	for(i=0;i<n*n;i++){
		kernel[i] = kernel[i]/suma;
	}
\end{lstlisting}
\vspace{1ex}
\normalsize
	Return: \\
\small
\begin{lstlisting}
	return kernel;
\end{lstlisting}
\vspace{1ex}
\normalsize
	\vspace{0.5cm}
	\Large{Function \texttt{free\_gaussian\_kernel}} \\
 
	\normalsize
	This function frees the memory allocated in the function \texttt{gaussian\_kernel}. It receives as parameter the pointer to the array to be freed. \\
\small
\begin{lstlisting}
void free_gaussian_kernel(double* kernel){
\end{lstlisting}
\vspace{1ex}
\normalsize
	Free memory: \\
\small
\begin{lstlisting}
	free(kernel);
\end{lstlisting}
\vspace{1ex}
\normalsize
	\vspace{0.5cm}
	\Large{Function \texttt{LoG\_kernel}} \\
 
	\normalsize
	This function generates a Laplacian of a Gaussian kernel (LoG kernel) of size $n\times n$ and standard deviation $\sigma$. \\
\small
\begin{lstlisting}
double *LoG_kernel(int n, float sigma){
\end{lstlisting}
\vspace{1ex}
\normalsize
	Memory allocation for the kernel: \\
\small
\begin{lstlisting}
	double *kernel = malloc(n*n*sizeof(double));
\end{lstlisting}
\vspace{1ex}
\normalsize
	We generate a Laplacian of a Gaussian kernel using the expression $\frac{x^2 + y^2 - 2\sigma^2}{\sigma^4}e^{-\frac{x^2+y^2}{2\sigma^2}}$. \\
\small
\begin{lstlisting}
	int i;
	int fila, col, x, y;
	double suma = 0;
	int imax = n*n;
	for(i=0;i<imax;i++){
		fila = (int)(i/n);
		col = i-(n*fila);
		y = ((int)(n/2))-fila;
		x = col-((int)(n/2));
		kernel[i] = ( (x*x + y*y - 2*sigma*sigma)/(sigma*sigma*sigma*sigma) )
					*exp(-(x*x + y*y)/(2*sigma*sigma));
	}
\end{lstlisting}
\vspace{1ex}
\normalsize
	Return: \\
\small
\begin{lstlisting}
	return kernel;
\end{lstlisting}
\vspace{1ex}
\normalsize


\subsection{2D convolution}
% Please do NOT edit this file.
% It has been automatically generated
% by a perl script from the original cxx sources
% in the Insight/Examples directory

% Any changes should be made in the file
% src/2dconvolution.c


The source code for this section can be found in the file \verb|2dconvolution.c|.\\  This file contains the necessary functions to compute the convolution between an 
  input array and a kernel (usually the input array is larger than the kernel). \\

	Includes:
\small
\begin{lstlisting}
	#include <stdlib.h> // malloc, calloc, free
	#include <stdio.h> // fprintf
\end{lstlisting}
\vspace{1ex}
\normalsize
	\vspace{0.5cm}
	\Large{Function \texttt{free\_gaussian\_kernel}} \\
 
	\normalsize
	This function frees the memory allocated in the function \texttt{get\_neighborhood}. It receives as parameter the pointer to the array to be freed. \\
\small
\begin{lstlisting}
void free_neighborhood(double* neighborhood){
\end{lstlisting}
\vspace{1ex}
\normalsize
	Free memory: \\
\small
\begin{lstlisting}
	free(neighborhood);
\end{lstlisting}
\vspace{1ex}
\normalsize
	\vspace{0.5cm}
	\Large{Function \texttt{free\_gaussian\_kernel}} \\
 
	\normalsize
	This function frees the memory allocated in the function \texttt{get\_neighbors\_offset}. It receives as parameter the pointer to the array to be freed. \\
\small
\begin{lstlisting}
void free_neighbors_offsets(int* offsets){
\end{lstlisting}
\vspace{1ex}
\normalsize
	Free memory: \\
\small
\begin{lstlisting}
	free(offsets);
\end{lstlisting}
\vspace{1ex}
\normalsize
	\vspace{0.5cm}
	\Large{Function \texttt{get\_neighbors\_offset}} \\
 
	\normalsize
	This function computes the linear array index offsets required to access the neighbor pixels in 
	a neighborhood of size $n\times n$. This computation requires to know the width $w$ of the image. \\
\small
\begin{lstlisting}
int *get_neighbors_offset(int w, int n){
\end{lstlisting}
\vspace{1ex}
\normalsize
	Memory allocation for the index offsets array: \\
\small
\begin{lstlisting}
	int *aux = malloc(n*n*sizeof(int));
\end{lstlisting}
\vspace{1ex}
\normalsize
	Computation of the index offsets array: \\
\small
\begin{lstlisting}
	int imax = n*n;
	int dif_fila_col = (n-1)/2;
	for (i=0;i<imax;i++){
		delta_fila = (int)(i/n)-dif_fila_col;
		delta_col = i-n*((int)(i/n))-dif_fila_col;
		aux[i] = delta_fila*w+delta_col;
	}
\end{lstlisting}
\vspace{1ex}
\normalsize
	Return: \\
\small
\begin{lstlisting}
	return aux;
\end{lstlisting}
\vspace{1ex}
\normalsize
	\vspace{0.5cm}
	\Large{Function \texttt{get\_neighborhood}} \\
 
	\normalsize
	This function returns an array of size $n\times n$ containing the neighbors of the 
  pixel at the position \texttt{pos} of the image \texttt{im}. For this calculation, 
	the index offsets array calculated with the function \texttt{get\_neighbors\_offset} 
	is needed.
\small
\begin{lstlisting}
double *get_neighborhood(double *im, int pos, int n, int* offsets){
\end{lstlisting}
\vspace{1ex}
\normalsize
	Memory allocation for the neighborhood array: \\
\small
\begin{lstlisting}
	double *aux = malloc(n*n*sizeof(double));
\end{lstlisting}
\vspace{1ex}
\normalsize
	Assigning values ​​of the neighborhood: \\
\small
\begin{lstlisting}
	int i;
	int imax = n*n;
	for (i=0;i<imax;i++){
		aux[i] = im[pos+offsets[i]];
	}
\end{lstlisting}
\vspace{1ex}
\normalsize
	Return: \\
\small
\begin{lstlisting}
	return aux;
\end{lstlisting}
\vspace{1ex}
\normalsize
	\vspace{0.5cm}
	\Large{Function \texttt{conv2d}} \\
 
	\normalsize
	This function calculates the convolution of image \texttt{input} (size $w\times h$) with the kernel \texttt{kernel} (size $n\times n$).
	Returns an image of size $(w+n-1)\times(h+n-1)$ (due to padding). The integer \texttt{padding\_method} manages the padding method: $0$
  means zero-padding, $1$ means image boundary reflection. \\
\small
\begin{lstlisting}
double *conv2d(double *input, int w, int h, double *kernel, int n, int padding_method){
\end{lstlisting}
\vspace{1ex}
\normalsize
	Zero padding: reserve memory for an array of size $(w+2(n-1))\times(h+2(n-1))$. \\
\small
\begin{lstlisting}
	int wx = (w+2*(n-1));
	int hx = (h+2*(n-1));
	double *aux = calloc(wx*hx,sizeof(double));
\end{lstlisting}
\vspace{1ex}
\normalsize
	Fill in the values ​​of the image \texttt{aux}, centering the original image \texttt{input} on it (zero-padding). \\
\small
\begin{lstlisting}
	int i,j,fila,col;
	int imax = wx*hx;
	if (padding_method == 0) {
		for(i=0;i<imax;i++){
			fila = (int)(i/wx);
			col = i-(wx*fila);
			if ( (fila>=n-1)&&(col>=n-1)&&(fila<h+n-1)&&(col<w+n-1) ) {
				aux[i] = input[(col-n+1)+(w*(fila-n+1))];
			}
		}
	}
\end{lstlisting}
\vspace{1ex}
\normalsize
 Other padding method: reflection of the image (if \texttt{padding\_method} equal to $1$). \\
\small
\begin{lstlisting}
	if (padding_method == 1) {
		int fila_refl, col_refl;
		for(i=0;i<imax;i++){
			fila = (int)(i/wx);
			col = i-(wx*fila);
			if (fila<n-1) {
				fila_refl = 2*n - 3 - fila;
				if (col<n-1) { //zone1
					col_refl = 2*n - 3 - col;
				} else if (col<w+n-1) {	//zone2
					col_refl = col;
				} else { //zone3
					col_refl = 2*w + 2*n - 3 - col;
				}
			} else if (fila<h+n-1) {
				fila_refl = fila;
				if (col<n-1) { //zone4
					col_refl = 2*n - 3 - col;
				} else if (col<w+n-1) { //image
					col_refl = col;
				} else { //zone5
					col_refl =  2*w + 2*n - 3 - col;
				}
			} else {
				fila_refl = 2*h + 2*n - 3 - fila;
				if (col<n-1) { //zone6
					col_refl =	2*n - 3 - col;
				} else if (col<w+n-1) {	//zone7
					col_refl = col;
				} else { //zone8
					col_refl =  2*w + 2*n - 3 - col;
				}
			}
			aux[i] = input[(col_refl-n+1)+(w*(fila_refl-n+1))];
		} //for
	} //if
\end{lstlisting}
\vspace{1ex}
\normalsize
	Reserve memory for the output array of size $(w+n-1)\times(h+n-1)$. \\
\small
\begin{lstlisting}
	double *out = malloc((w+n-1)*(h+n-1)*sizeof(double));
\end{lstlisting}
\vspace{1ex}
\normalsize
	Compute the convolution. Most of the operations are intended to calculate the relative 
	positions between the images \texttt{aux} and \texttt{out}. \\
\small
\begin{lstlisting}
	double acum = 0;
	int pos;
	// Convolution
	imax = (w+n-1)*(h+n-1);
	int jmax = n*n;
	int *offsets = get_neighbors_offset(wx, n);
	int dif_fila_col = (n-1)/2;
	for(i=0;i<imax;i++){
		fila = (int)(i/(w+n-1));
		col = i-((w+n-1)*fila);
		fila += dif_fila_col;
		col += dif_fila_col;
		pos = wx*fila + col;
		// compute convolution:
		acum = 0;
		for (j=0;j<jmax;j++){
			acum += aux[pos+offsets[j]]*kernel[j];
		}
		out[i] = acum;
	}
\end{lstlisting}
\vspace{1ex}
\normalsize
	Free and return: \\
\small
\begin{lstlisting}
	free(aux);
	free_neighbors_offsets(offsets);
	return out;
	//return aux; //TEST
\end{lstlisting}
\vspace{1ex}
\normalsize


\end{comment}

%-------------------------------------------------------------------------------

\section{Results}
\label{sec:results}

The results of each algorithm are first shown in a simple case, where a test image composed of a white square on a black background (shown in Figure \ref{fig:original1}) is used. 

\begin{figure}[h!]
	\centering
	\includegraphics[width=0.2\textwidth]{results/square127.png}
	\caption{Test image 1: white square on black background (127$\times$127 pixels).}
	\label{fig:original1}
\end{figure}

Figures \ref{fig:result1-a} to \ref{fig:result1-f} show the output of each algorithm. The execution times\footnote{Running on Intel Core i3 CPU (2.53GHz), 3 Gb RAM, standard laptop.} for each algorithm are shown in Table \ref{exectime1}. 

\begin{figure}[h!]
	\centering
	\subfigure[Roberts. $th=0.1$.]{\label{fig:result1-a}\includegraphics[width=0.2\textwidth]{results/square127_roberts.png}}
	\quad
	\subfigure[Prewitt. $th=0.1$.]{\label{fig:result1-b}\includegraphics[width=0.2\textwidth]{results/square127_prewitt.png}}
	\quad
	\subfigure[Sobel. $th=0.1$.]{\label{fig:result1-c}\includegraphics[width=0.2\textwidth]{results/square127_sobel.png}}

	\subfigure[Marr-Hildreth (Gaussian). $\sigma=1.5$, $n=13$, $th_{ZC}=0.1$.]{\label{fig:result1-d}\includegraphics[width=0.2\textwidth]{results/square127_marr-hildreth.png}}
	\quad
	\subfigure[Marr-Hildreth (LoG). $\sigma=1.5$, $n=17$, $th_{ZC}=0.1$.]{\label{fig:result1-e}\includegraphics[width=0.2\textwidth]{results/square127_marr-hildreth-log.png}}
	\quad
	\subfigure[Haralick. $\rho=0.4$.]{\label{fig:result1-f}\includegraphics[width=0.2\textwidth]{results/square127_haralick.png}}
	\caption{Results of the algorithms using the Figure \ref{fig:original1} as input image. The first three (\subref{fig:result1-a}, \subref{fig:result1-b} and \subref{fig:result1-c}) correspond to the first derivative methods (Roberts, Prewitt and Sobel), then the following two (\subref{fig:result1-d} and \subref{fig:result1-e}) are from the Marr-Hildreth algorithm using Gaussian and LoG kernels, and the last one (\subref{fig:result1-f}) corresponds to the Haralick algorithm. Note the rounded corners on the Marr-Hildreth results, due to Gaussian blur.}
	\label{fig:result1}
\end{figure}

\begin{table}[h!]
	\begin{center}
	\begin{tabular}{| l | r |}
		\hline \rule{0pt}{3ex}
		\cellcolor[gray]{0.8} \textbf{Algorithm}	& \cellcolor[gray]{0.8} \textbf{Execution time (s)}	\\ \hline \rule{0pt}{3ex}
		Roberts, Prewitt and Sobel					& $0.020 \ s$										\\ \hline \rule{0pt}{3ex}
		Marr-Hildreth (Gaussian)					& $0.030 \ s$										\\ \hline \rule{0pt}{3ex}
		Marr-Hildreth (LoG)							& $0.050 \ s$										\\ \hline \rule{0pt}{3ex}
		Haralick									& $0.040 \ s$										\\
		\hline
	\end{tabular}
	\end{center}
	\caption{Execution time of the algorithms (including I/O) using the Figure \ref{fig:original1} as input image (127$\times$127 pixels, 1 channel).}
	\label{exectime1}
\end{table}
\vspace{0.5cm}

It is appreciated that in the simplest cases, the first derivative edge detection algorithms are running better and faster. All algorithms show consistent results. Thicker edges appear in the results of Haralick, due to smooth image model that is imposed in this algorithm. 

Now the performance of each algorithm is analyzed, using a natural image shown in Figure \ref{fig:original2}. 

\begin{figure}[t!]
	\centering
	\includegraphics[width=0.48\textwidth]{results/molino_bw.png}
	\caption{Test image 2: windmill (1000$\times$563 pixels).}
	\label{fig:original2}
\end{figure}

First the results of the first derivative edge detection algorithms is shown Figures \ref{fig:result2-a} to \ref{fig:result2-c}. Figures \ref{fig:result2-d} and \ref{fig:result2-e} show the results obtained using the Marr-Hildreth algorithm (both Gaussian and LoG kernels), and Figure \ref{fig:result2-f} shows the results obtained using the Haralick algorithm. Table \ref{exectime2} shows the execution time for each of the algorithms. 

\begin{figure}[h!]
	\centering
	\subfigure[Roberts. $th=0.1$.]{\label{fig:result2-a}\includegraphics[width=0.42\textwidth]{results/molino_roberts.png}}
	\quad
	\subfigure[Prewitt. $th=0.1$.]{\label{fig:result2-b}\includegraphics[width=0.42\textwidth]{results/molino_prewitt.png}}
	
	\subfigure[Sobel. $th=0.1$.]{\label{fig:result2-c}\includegraphics[width=0.42\textwidth]{results/molino_sobel.png}}
	\quad
	\subfigure[Marr-Hildreth (Gaussian). $\sigma=3$, $n=25$, $th_{ZC}=0.07$.]{\label{fig:result2-d}\includegraphics[width=0.42\textwidth]{results/molino_marr-hildreth.png}}

	\subfigure[Marr-Hildreth (LoG). $\sigma=3$, $n=29$, $th_{ZC}=0.13$.]{\label{fig:result2-e}\includegraphics[width=0.42\textwidth]{results/molino_marr-hildreth-log.png}}
	\quad
	\subfigure[Haralick. $\rho=0.4$.]{\label{fig:result2-f}\includegraphics[width=0.42\textwidth]{results/molino_haralick.png}}
	\caption{Results of the first derivative, Marr-Hildreth and Haralick's algorithms using Figure \ref{fig:original2} as input image.}
	\label{fig:result2}
\end{figure}

\begin{table}[t!]
	\begin{center}
	\begin{tabular}{| l | r |}
		\hline \rule{0pt}{3ex}
		\cellcolor[gray]{0.8} \textbf{Algorithm}	& \cellcolor[gray]{0.8} \textbf{Execution time (s)}	\\ \hline \rule{0pt}{3ex}
		Roberts, Prewitt and Sobel					& $0.550 \ s$										\\ \hline \rule{0pt}{3ex}
		Marr-Hildreth (Gaussian)					& $1.050 \ s$										\\ \hline \rule{0pt}{3ex}
		Marr-Hildreth (LoG)							& $1.440 \ s$										\\ \hline \rule{0pt}{3ex}
		Haralick									& $0.930 \ s$										\\
		\hline
	\end{tabular}
	\end{center}
	\caption{Execution time of the algorithms (including I/O) using Figure \ref{fig:original2} as input image (1000$\times$563 pixels, 3 channels).}
	\label{exectime2}
\end{table}

In this example the difference in performance between the first derivative edge detection algorithms and the second derivative ones is more significative. It can be seen in Figure \ref{fig:result3} an area of ​​interest of the image, enlarged to have a better view of the detail. Note that none of the first derivative methods (even with a loose threshold) detect the lower edge of the blade (which is shaded). Neither the Haralick algorithm is able to detect it. However, the Marr-Hildreth algorithm detects it, using either a Gaussian or a LoG kernel. 

Another observation is that edges detected by Haralick's algorithm appear to be thicker than those detected with other ones. This may be due the regularity that Haralick assumes. 

\begin{figure}[t!]
	\centering
	\subfigure[Original.]{\label{fig:result3-a}\includegraphics[width=0.48\textwidth]{results/molino_bw_zoom.png}}

	\subfigure[Roberts.]{\label{fig:result3-b}\includegraphics[width=0.48\textwidth]{results/molino_roberts_zoom.png}}
	\quad
	\subfigure[Prewitt.]{\label{fig:result3-c}\includegraphics[width=0.48\textwidth]{results/molino_prewitt_zoom.png}}
	
	\subfigure[Sobel.]{\label{fig:result3-d}\includegraphics[width=0.48\textwidth]{results/molino_sobel_zoom.png}}
	\quad
	\subfigure[Marr-Hildreth (Gaussian).]{\label{fig:result3-e}\includegraphics[width=0.48\textwidth]{results/molino_marr-hildreth_zoom.png}}

	\subfigure[Marr-Hildreth (LoG).]{\label{fig:result3-f}\includegraphics[width=0.48\textwidth]{results/molino_marr-hildreth-log_zoom.png}}
	\quad
	\subfigure[Haralick.]{\label{fig:result3-g}\includegraphics[width=0.48\textwidth]{results/molino_haralick_zoom.png}}
	\caption{Results of the algorithms using Figure \ref{fig:original2} as input image, enlarging an interesting area.}
	\label{fig:result3}
\end{figure}

%-------------------------------------------------------------------------------

\subsection{Further examples}
\label{sec:examples}

Further examples obtained with the implemented algorithms are shown in Figures \ref{fig:example1} and \ref{fig:example2}. More examples can be found in the online \href{http://dev.ipol.im/~haldos/ipol_demo/xxx_edges/}{demo}.

\begin{figure}[h!]
	\centering
	\subfigure[Original.]{\includegraphics[width=0.3\textwidth]{examples/lena_bw.png}}

	\subfigure[Roberts. $th=0.1$.]{\includegraphics[width=0.3\textwidth]{examples/lena_roberts.png}}
	\quad
	\subfigure[Prewitt. $th=0.1$.]{\includegraphics[width=0.3\textwidth]{examples/lena_prewitt.png}}
	\quad
	\subfigure[Sobel. $th=0.1$.]{\includegraphics[width=0.3\textwidth]{examples/lena_sobel.png}}

	\subfigure[Marr-Hildreth (Gaussian). $\sigma=2$, $n=21$, $th_{ZC}=0.15$.]{\includegraphics[width=0.3\textwidth]{examples/lena_marr-hildreth.png}}
	\quad
	\subfigure[Marr-Hildreth (LoG). $\sigma=2$, $n=25$, $th_{ZC}=0.15$.]{\includegraphics[width=0.3\textwidth]{examples/lena_marr-hildreth-log.png}}
	\quad
	\subfigure[Haralick. $\rho=0.35$.]{\includegraphics[width=0.3\textwidth]{examples/lena_haralick.png}}
	\caption{Example: Lena (512$\times$512 pixels).}
	\label{fig:example1}
\end{figure}

Some observations: 
\begin{itemize}
	\item The first derivative algorithms, although they work properly and quickly, have some problems like discontinuity of the edges. They are also very affected by noise in images, appearing unwanted isolated edges (More sophisticated methods help to avoid and improve this, e.g. Canny edge detector \cite{Canny1986}). 
	\item The Marr-Hildreth algorithm (in its two versions) achieved interesting results in the edge detail. This algorithm provides a first attempt to obtain regular edges, because of the filtering with a Gaussian kernel. This behavior can be seen in Lena's hair (Figure \ref{fig:example1}), or inside the oranges (example \ref{fig:example2}). 
	\item As mentioned earlier, Haralick's algorithm is the first to assume higher degree of regularity in the neighborhood of a pixel (third order). This causes thicker edges, as shown in both examples. But, note that some edges are only detected with this algorithm, e.g. the lower edges of the oranges, which are in the shadow, in Figure \ref{fig:example2}. These edges are quite smooth, so that the Haralick model is well suited to them, and they do not represent an abrupt change in the intensity to be detected by other methods. 
\end{itemize}

\begin{figure}[t!]
	\centering
	\subfigure[Original.]{\includegraphics[width=0.3\textwidth]{examples/oranges_bw.png}}

	\subfigure[Roberts. $th=0.1$]{\includegraphics[width=0.3\textwidth]{examples/oranges_roberts.png}}
	\quad
	\subfigure[Prewitt. $th=0.1$]{\includegraphics[width=0.3\textwidth]{examples/oranges_prewitt.png}}
	\quad
	\subfigure[Sobel. $th=0.1$]{\includegraphics[width=0.3\textwidth]{examples/oranges_sobel.png}}

	\subfigure[Marr-Hildreth (Gaussian). $\sigma=2$, $n=21$, $th_{ZC}=0.15$.]{\includegraphics[width=0.3\textwidth]{examples/oranges_marr-hildreth.png}}
	\quad
	\subfigure[Marr-Hildreth (LoG). $\sigma=2$, $n=25$, $th_{ZC}=0.15$.]{\includegraphics[width=0.3\textwidth]{examples/oranges_marr-hildreth-log.png}}
	\quad
	\subfigure[Haralick. $\rho=0.4$.]{\includegraphics[width=0.3\textwidth]{examples/oranges_haralick.png}}
	\caption{Example: Oranges (536$\times$480 pixels).}
	\label{fig:example2}
\end{figure}


%-------------------------------------------------------------------------------

\subsection{Video}

This is an example of applying the edge detection algorithms, frame by frame, to a video: 
%\begin{itemize}
%	\centering
%	\item \href{http://iie.fing.edu.uy/~haldos/ipol/video.mov}{original} (43 Mb).
%\end{itemize}
\begin{multicols}{2}
\begin{itemize}
	\item \href{http://iie.fing.edu.uy/~haldos/ipol/video.mov}{original} (43 Mb).
	\item \href{http://iie.fing.edu.uy/~haldos/ipol/video-roberts-wide_0.1.mov}{roberts version} (7.9 Mb).
	\item \href{http://iie.fing.edu.uy/~haldos/ipol/video-prewitt-wide_0.1.mov}{prewitt version} (9.1 Mb).
	\item \href{http://iie.fing.edu.uy/~haldos/ipol/video-sobel-wide_0.1.mov}{sobel version} (9.1 Mb).
	\item \href{http://iie.fing.edu.uy/~haldos/ipol/video-marr-hildreth-gaussian-wide_3_19_0.04.mov}{marr-hildreth-gaussian version} (12 Mb).
	\item \href{http://iie.fing.edu.uy/~haldos/ipol/video-marr-hildreth-log-wide_3_25_0.04.mov}{marr-hildreth-log version} (14 Mb).
	\item \href{http://iie.fing.edu.uy/~haldos/ipol/video-haralick-wide_0.5.mov}{haralick version} (12 Mb).
\end{itemize}
\end{multicols}

\clearpage
%-------------------------------------------------------------------------------

\section{Conclusions}
\label{sec:conclusions}

Some of the most traditional methods of edge detection in digital images were discussed and carefully implemented in this work. The implemented algorithms were tested with synthetic and real images, obtaining generally the expected results, taking into account the limitations of these methods. 

The first derivative algorithms (Roberts, Prewitt and Sobel) have the advantage of having a very simple implementation. Also they run extremely fast, because they only consists of a convolution with a very small kernel (2$\times$2 or 3$\times$3 pixels). The results obtained with these methods are quite good, considering their simplicity, but they have problems such as noise and discontinuity of the edges. Gaussian filtering would reduce noise, alleviate (but not solve) the discontinuity of the edges, and would make more fair the comparison with the second derivative methods. 

The second derivative algoritms (Marr-Hildreth \& Haralick) involve several more operations, since more convolutions (and with larger kernels) are performed. In real images, they have better behavior than first derivative algorithms. 

Comparing the two versions of the Marr-Hildreth algorithm, the version with Gaussian kernel runs significative faster that the LoG one. The latter, while slower (since it needs a larger kernel to achieve similar results) is more accurate, because it makes no approximation to calculate the Laplacian (it is calculated analytically before creating the kernel). It is also possible to manage the size of the kernel, which represents a scale parameter of the algorithm, being able to obtain a highly detailed edge image using small kernels, or just more noticeable edges using larger kernels. 

Haralick's algorithm, although it shares the Marr-Hildreth idea of finding zero crossings of the second derivative, has some quite different ideas. It is the only one of these algorithms which works with an approximation of the intensity of the image in the neighborhood of a pixel, using a bicubic polinomial function. This supposes a certain regularity in the image, which is not always true, and sometimes leads to detect edges where none of them exist, and get some thicker edges too. 

All algorithms have their advantages and disadvantages. Choosing one or the other may depend on requirements of the application. Furthermore, none of the implemented methods ensure edge connectivity, as the Canny \cite{Canny1986} algorithm does.

%To conclude, this paper is a summary of some classical edge detection algorithms. In the preparation of this paper, no similar material (concentrated in one text) was found, so this paper appears to be useful material for whom starts studying edge detection methods. \\

As an addendum to this work a detailed implementation of the described algorithms is presented, 
available in C language, along with an online demo where users can run the algorithms 
with arbitrary images uploaded by themselves.

\clearpage

%-------------------------------------------------------------------------------

\bibliography{bibliography}

%-------------------------------------------------------------------------------

\end{document}
