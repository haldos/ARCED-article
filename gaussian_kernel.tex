% Please do NOT edit this file.
% It has been automatically generated
% by a perl script from the original cxx sources
% in the Insight/Examples directory

% Any changes should be made in the file
% src/gaussian_kernel.c


The source code for this section can be found in the file \verb|gaussian_kernel.c|.\\	This file implements the necessary functions for generating 
	Gaussian and LoG (Laplacian of a Gaussian) kernels. It is also 
	done here the alloc and free of the memory needed. \\

	Includes:
\small
\begin{lstlisting}
	#include <math.h> // exp
	#include <stdlib.h> // malloc, calloc, free
	#include <stdio.h> // fprintf
\end{lstlisting}
\vspace{1ex}
\normalsize
	\vspace{0.5cm}
	\Large{Function \texttt{gaussian\_kernel}} \\
 
	\normalsize
	This function generates a Gaussian kernel of size $n\times n$ and standard deviation $\sigma$. \\
\small
\begin{lstlisting}
double *gaussian_kernel(int n, float sigma){
\end{lstlisting}
\vspace{1ex}
\normalsize
	Memory allocation for the kernel: \\
\small
\begin{lstlisting}
	double *kernel = malloc(n*n*sizeof(double));
\end{lstlisting}
\vspace{1ex}
\normalsize
	A normalized Gaussian kernel is generated using the expression $e^{-\frac{x^2+y^2}{2\sigma^2}}$. \\
\small
\begin{lstlisting}
	int i;
	int fila, col, x, y;
	double suma = 0;
	int imax = n*n;
	for(i=0;i<imax;i++){
		fila = (int)(i/n);
		col = i-(n*fila);
		y = ((int)(n/2))-fila;
		x = col-((int)(n/2));
		kernel[i] = exp(-(x*x + y*y)/(2*sigma*sigma));
		suma += kernel[i];
	}
\end{lstlisting}
\vspace{1ex}
\normalsize
	Kernel normalization, using the sum of its components. \\
\small
\begin{lstlisting}
	for(i=0;i<n*n;i++){
		kernel[i] = kernel[i]/suma;
	}
\end{lstlisting}
\vspace{1ex}
\normalsize
	Return: \\
\small
\begin{lstlisting}
	return kernel;
\end{lstlisting}
\vspace{1ex}
\normalsize
	\vspace{0.5cm}
	\Large{Function \texttt{free\_gaussian\_kernel}} \\
 
	\normalsize
	This function frees the memory allocated in the function \texttt{gaussian\_kernel}. It receives as parameter the pointer to the array to be freed. \\
\small
\begin{lstlisting}
void free_gaussian_kernel(double* kernel){
\end{lstlisting}
\vspace{1ex}
\normalsize
	Free memory: \\
\small
\begin{lstlisting}
	free(kernel);
\end{lstlisting}
\vspace{1ex}
\normalsize
	\vspace{0.5cm}
	\Large{Function \texttt{LoG\_kernel}} \\
 
	\normalsize
	This function generates a Laplacian of a Gaussian kernel (LoG kernel) of size $n\times n$ and standard deviation $\sigma$. \\
\small
\begin{lstlisting}
double *LoG_kernel(int n, float sigma){
\end{lstlisting}
\vspace{1ex}
\normalsize
	Memory allocation for the kernel: \\
\small
\begin{lstlisting}
	double *kernel = malloc(n*n*sizeof(double));
\end{lstlisting}
\vspace{1ex}
\normalsize
	We generate a Laplacian of a Gaussian kernel using the expression $\frac{x^2 + y^2 - 2\sigma^2}{\sigma^4}e^{-\frac{x^2+y^2}{2\sigma^2}}$. \\
\small
\begin{lstlisting}
	int i;
	int fila, col, x, y;
	double suma = 0;
	int imax = n*n;
	for(i=0;i<imax;i++){
		fila = (int)(i/n);
		col = i-(n*fila);
		y = ((int)(n/2))-fila;
		x = col-((int)(n/2));
		kernel[i] = ( (x*x + y*y - 2*sigma*sigma)/(sigma*sigma*sigma*sigma) )
					*exp(-(x*x + y*y)/(2*sigma*sigma));
	}
\end{lstlisting}
\vspace{1ex}
\normalsize
	Return: \\
\small
\begin{lstlisting}
	return kernel;
\end{lstlisting}
\vspace{1ex}
\normalsize
