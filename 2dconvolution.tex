% Please do NOT edit this file.
% It has been automatically generated
% by a perl script from the original cxx sources
% in the Insight/Examples directory

% Any changes should be made in the file
% src/2dconvolution.c


The source code for this section can be found in the file \verb|2dconvolution.c|.\\  This file contains the necessary functions to compute the convolution between an 
  input array and a kernel (usually the input array is larger than the kernel). \\

	Includes:
\small
\begin{lstlisting}
	#include <stdlib.h> // malloc, calloc, free
	#include <stdio.h> // fprintf
\end{lstlisting}
\vspace{1ex}
\normalsize
	\vspace{0.5cm}
	\Large{Function \texttt{free\_gaussian\_kernel}} \\
 
	\normalsize
	This function frees the memory allocated in the function \texttt{get\_neighborhood}. It receives as parameter the pointer to the array to be freed. \\
\small
\begin{lstlisting}
void free_neighborhood(double* neighborhood){
\end{lstlisting}
\vspace{1ex}
\normalsize
	Free memory: \\
\small
\begin{lstlisting}
	free(neighborhood);
\end{lstlisting}
\vspace{1ex}
\normalsize
	\vspace{0.5cm}
	\Large{Function \texttt{free\_gaussian\_kernel}} \\
 
	\normalsize
	This function frees the memory allocated in the function \texttt{get\_neighbors\_offset}. It receives as parameter the pointer to the array to be freed. \\
\small
\begin{lstlisting}
void free_neighbors_offsets(int* offsets){
\end{lstlisting}
\vspace{1ex}
\normalsize
	Free memory: \\
\small
\begin{lstlisting}
	free(offsets);
\end{lstlisting}
\vspace{1ex}
\normalsize
	\vspace{0.5cm}
	\Large{Function \texttt{get\_neighbors\_offset}} \\
 
	\normalsize
	This function computes the linear array index offsets required to access the neighbor pixels in 
	a neighborhood of size $n\times n$. This computation requires to know the width $w$ of the image. \\
\small
\begin{lstlisting}
int *get_neighbors_offset(int w, int n){
\end{lstlisting}
\vspace{1ex}
\normalsize
	Memory allocation for the index offsets array: \\
\small
\begin{lstlisting}
	int *aux = malloc(n*n*sizeof(int));
\end{lstlisting}
\vspace{1ex}
\normalsize
	Computation of the index offsets array: \\
\small
\begin{lstlisting}
	int imax = n*n;
	int dif_fila_col = (n-1)/2;
	for (i=0;i<imax;i++){
		delta_fila = (int)(i/n)-dif_fila_col;
		delta_col = i-n*((int)(i/n))-dif_fila_col;
		aux[i] = delta_fila*w+delta_col;
	}
\end{lstlisting}
\vspace{1ex}
\normalsize
	Return: \\
\small
\begin{lstlisting}
	return aux;
\end{lstlisting}
\vspace{1ex}
\normalsize
	\vspace{0.5cm}
	\Large{Function \texttt{get\_neighborhood}} \\
 
	\normalsize
	This function returns an array of size $n\times n$ containing the neighbors of the 
  pixel at the position \texttt{pos} of the image \texttt{im}. For this calculation, 
	the index offsets array calculated with the function \texttt{get\_neighbors\_offset} 
	is needed.
\small
\begin{lstlisting}
double *get_neighborhood(double *im, int pos, int n, int* offsets){
\end{lstlisting}
\vspace{1ex}
\normalsize
	Memory allocation for the neighborhood array: \\
\small
\begin{lstlisting}
	double *aux = malloc(n*n*sizeof(double));
\end{lstlisting}
\vspace{1ex}
\normalsize
	Assigning values ​​of the neighborhood: \\
\small
\begin{lstlisting}
	int i;
	int imax = n*n;
	for (i=0;i<imax;i++){
		aux[i] = im[pos+offsets[i]];
	}
\end{lstlisting}
\vspace{1ex}
\normalsize
	Return: \\
\small
\begin{lstlisting}
	return aux;
\end{lstlisting}
\vspace{1ex}
\normalsize
	\vspace{0.5cm}
	\Large{Function \texttt{conv2d}} \\
 
	\normalsize
	This function calculates the convolution of image \texttt{input} (size $w\times h$) with the kernel \texttt{kernel} (size $n\times n$).
	Returns an image of size $(w+n-1)\times(h+n-1)$ (due to padding). The integer \texttt{padding\_method} manages the padding method: $0$
  means zero-padding, $1$ means image boundary reflection. \\
\small
\begin{lstlisting}
double *conv2d(double *input, int w, int h, double *kernel, int n, int padding_method){
\end{lstlisting}
\vspace{1ex}
\normalsize
	Zero padding: reserve memory for an array of size $(w+2(n-1))\times(h+2(n-1))$. \\
\small
\begin{lstlisting}
	int wx = (w+2*(n-1));
	int hx = (h+2*(n-1));
	double *aux = calloc(wx*hx,sizeof(double));
\end{lstlisting}
\vspace{1ex}
\normalsize
	Fill in the values ​​of the image \texttt{aux}, centering the original image \texttt{input} on it (zero-padding). \\
\small
\begin{lstlisting}
	int i,j,fila,col;
	int imax = wx*hx;
	if (padding_method == 0) {
		for(i=0;i<imax;i++){
			fila = (int)(i/wx);
			col = i-(wx*fila);
			if ( (fila>=n-1)&&(col>=n-1)&&(fila<h+n-1)&&(col<w+n-1) ) {
				aux[i] = input[(col-n+1)+(w*(fila-n+1))];
			}
		}
	}
\end{lstlisting}
\vspace{1ex}
\normalsize
 Other padding method: reflection of the image (if \texttt{padding\_method} equal to $1$). \\
\small
\begin{lstlisting}
	if (padding_method == 1) {
		int fila_refl, col_refl;
		for(i=0;i<imax;i++){
			fila = (int)(i/wx);
			col = i-(wx*fila);
			if (fila<n-1) {
				fila_refl = 2*n - 3 - fila;
				if (col<n-1) { //zone1
					col_refl = 2*n - 3 - col;
				} else if (col<w+n-1) {	//zone2
					col_refl = col;
				} else { //zone3
					col_refl = 2*w + 2*n - 3 - col;
				}
			} else if (fila<h+n-1) {
				fila_refl = fila;
				if (col<n-1) { //zone4
					col_refl = 2*n - 3 - col;
				} else if (col<w+n-1) { //image
					col_refl = col;
				} else { //zone5
					col_refl =  2*w + 2*n - 3 - col;
				}
			} else {
				fila_refl = 2*h + 2*n - 3 - fila;
				if (col<n-1) { //zone6
					col_refl =	2*n - 3 - col;
				} else if (col<w+n-1) {	//zone7
					col_refl = col;
				} else { //zone8
					col_refl =  2*w + 2*n - 3 - col;
				}
			}
			aux[i] = input[(col_refl-n+1)+(w*(fila_refl-n+1))];
		} //for
	} //if
\end{lstlisting}
\vspace{1ex}
\normalsize
	Reserve memory for the output array of size $(w+n-1)\times(h+n-1)$. \\
\small
\begin{lstlisting}
	double *out = malloc((w+n-1)*(h+n-1)*sizeof(double));
\end{lstlisting}
\vspace{1ex}
\normalsize
	Compute the convolution. Most of the operations are intended to calculate the relative 
	positions between the images \texttt{aux} and \texttt{out}. \\
\small
\begin{lstlisting}
	double acum = 0;
	int pos;
	// Convolution
	imax = (w+n-1)*(h+n-1);
	int jmax = n*n;
	int *offsets = get_neighbors_offset(wx, n);
	int dif_fila_col = (n-1)/2;
	for(i=0;i<imax;i++){
		fila = (int)(i/(w+n-1));
		col = i-((w+n-1)*fila);
		fila += dif_fila_col;
		col += dif_fila_col;
		pos = wx*fila + col;
		// compute convolution:
		acum = 0;
		for (j=0;j<jmax;j++){
			acum += aux[pos+offsets[j]]*kernel[j];
		}
		out[i] = acum;
	}
\end{lstlisting}
\vspace{1ex}
\normalsize
	Free and return: \\
\small
\begin{lstlisting}
	free(aux);
	free_neighbors_offsets(offsets);
	return out;
	//return aux; //TEST
\end{lstlisting}
\vspace{1ex}
\normalsize
