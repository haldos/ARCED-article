\documentclass[a4paper,10pt]{report}

\usepackage[english]{babel}
%\usepackage[none]{hyphenat}
\usepackage{graphics}
\usepackage{graphicx}
\usepackage[pdftex]{hyperref}
%\usepackage[tight,scriptsize]{subfigure}
%\usepackage{ucs}
%\usepackage[utf8x]{inputenc}
\usepackage{jfc}
\hypersetup{hypertexnames, plainpages=false,colorlinks=true}
\usepackage{amsmath}

% Title Page
\title{}
\author{}

\usepackage{xr}
\externaldocument{../../edge-detection-review-3rd-revision-v3}
\reversemarginpar

\newcounter{myquestion}
\newcommand{\ml}{\mathcal{L}}
\newcommand{\ms}{\mathcal{S}}

\begin{document}
\maketitle

%\begin{abstract}
%\end{abstract}

\chapter{Introduction}
In this communication we answer to the fourth round of review of the article "A review of Classic edge detectors", submitted to the Image Processing Online Journal.
In the first place, we would like to thank again the reviewers for the detailed and thoughtful review provided. In the following we address the issues raised by the reviewers. For each issue, an answer is provided and a cross-reference with a red tag $C_n$ (where n is the number of the correction) to the corresponding (corrected) part of the paper is provided. In addition, changes made to the paper are highlighted in red (when significative) in the new version of the paper. 
These red marks will be removed for the final version.

The remarks from the reviewers are marked with the tag \textcolor{blue}{QUESTION} and our answers with the tag \ans.

%\section{Brief summary of changes}
\section{General changes}
\begin{itemize}
 \item Improved instructions in makefile to compile in OSX (libpng headers not found)
 \item Quality of all synthetic images (kernels) has been improved (See corrections \ref{math:common:gauss},\ref{math:common:log} and \ref{math:common:comparison})

\end{itemize}


\chapter{Answers to reviewer \#1}

\que I think the paper scientific content is now ready.\\

\que Apart from that, I had some problems with an experiment (see below) and I
think the authors did not follow the reviewers advice to reread the paper.
There were numerous grammatical errors, as well as typos or presentation
problems, which could have been easily avoided and made this review very
painful.
I'll leave it to the editor to decide whether he prefers another revision or
move directly to editing.\\

\section{CODE}

\begin{verbatim}

The code compiles fine, and "make test" also works. But when I run it with
"oranges.png" it fails (tried several parameters and both prewitt and
roberts). The problem seems to happen when reading the image

 ./edges input/oranges.png -p 0.2
A review of classic edge detection algorithms
Haldo Sponton & Juan Cardelino, IPOL 2014

Input image: 0.2
Size: 140655700885148 x 140733592255632

Number of algorithms to run: 1
Error: xmalloc: out of memory
Running Prewitt, threshold=0.20"

On the other hand, the image size is ok when I don't try to run any
algorithm:
 ./edges input/oranges.png
A review of classic edge detection algorithms
Haldo Sponton & Juan Cardelino, IPOL 2014

Input image: input/oranges.png
Size: 536 x 480

Number of algorithms to run: 0



Can you check it out?

\end{verbatim}

\ans the problem is related to the order of the arguments. The input images must be the last parameter in every case. I noted that on the README.txt.\\

\que Concerning my requirement for the demo, I was not asking for an archive of
all demo results, which would be meaningless, but for each experiment an
archive containing the results of  the different detections instead of
having to download each detection result separately.\\

\ans I agree it would be useful, but I propose to leave it to a future version of the demo. I think this can be done in the IPOL framework without another round of review.\\

\section{Paper}

\que Algorithm 2 why not merge lines 7-8?\\

\ans done.\\

Then a list of errors (again). AGAIN I did not write all mistakes down, so I
strongly encourage you to proofread your paper.\\

\que abstract: results obtained from a qualitative point of view... -> rephrase\\
\ans rephrased.\\

\que p1: "operations operations"\\
\ans done\\

\que p1: two footnotes with the same url address.\\
\ans corrected.\\

\que p2: "a disconnected an unordered "???\\
\ans corrected.\\

\que p4:  a slight variation of [...] difference -> rephrase, no main verb in
that sentence.\\
\ans rephrased.\\

\que p4: "their filter are" -> singular or plural?\\
\ans corrected.\\

\que p4: different thresholds lead to different results. Thicker... -> results:
thicker...\\
\ans corrected.\\

\que p5: [h!] in front of each figure.\\
\ans corrected.\\

\que p8: consists on convolving\\
\ans corrected.\\

\que p8 not only... BUT also\\
\ans rephrased.\\

\que p24: "are arise"\\
\ans corrected.\\

\que p25: ".\\
\ans corrected.\\

\que p29: "results [...] algorithms our sample images." fig 20-21-22-23\\
\ans corrected.\\

\chapter{Answers to reviewer \#2}

The comments of this reviewer were addressed in the previous version.\\

\bibliographystyle{unsrt}
\bibliography{../../tools/journal_list_short,../../tools/references,../../tools/hierarchical_segmentation}
\end{document}


 
